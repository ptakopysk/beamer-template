\documentclass[handout,aspectratio=169]{beamer}

\usepackage[english]{babel}
\usepackage{ufalslides}
\usepackage{xcolor}
\usepackage{textcomp}

% %%%%%%%%%%%%%%%%%%%%%%%%%%%%%%%%%%%%%%%%%%%%%%%%%%%%%%%%%%%%%%%%%%%%%%%%%%%%%
\def\course{NPFL000 Name of the course}
\def\courseurl{https://ufal.cz/courses/npfl000}
\def\title{\LaTeX~template for LangTech courses taught at ÚFAL}
\def\author{Jindřich Libovický}
\def\date{September 7, 2018}
\def\licence{cc-by-nc-sa}
\def\langtech{}
\def\shownavigation{}
% %%%%%%%%%%%%%%%%%%%%%%%%%%%%%%%%%%%%%%%%%%%%%%%%%%%%%%%%%%%%%%%%%%%%%%%%%%%%%

\begin{document}

\maketitle

\outline{Outline}

\section{A section}
\outlinecurrent{Outline}

\begin{frame}[fragile]
    \frametitle{How to use the template}

    \begin{lstlisting}[language=TeX]
\documentclass[handout,aspectratio=169]{beamer}
\usepackage[english]{babel}
\usepackage{ufalslides}
\end{lstlisting}

    \begin{itemize}
        \item Use \lstinline{handout} option if you want to generate a handout without
    animations.

        \item Before you begin document, define what you want to
            appear in the title slide (see the next slide for more info).
    \end{itemize}

\end{frame}


\begin{frame}[fragile]
    \frametitle{Content of the title page}

1. Define the content of the title page
    \begin{lstlisting}[language=TeX]
\def\course{NPFL116 Compendium of Neural Machine Translation}
\def\courseurl{https://ufal.cz/courses/npfl000}
\def\title{Attention Mechanism}
\def\subtitle{How to attend with a mechanism}
\def\author{Jindřich Libovický, Jindřich Helcl} \def\date{March 1, 2017}
\def\licence{cc-by-nc-sa}
\def\langtech{} % shows the LangTech and the EU logo
\def\shownavigation{} % shows the navigation links in the bottom line
\end{lstlisting}

    {\tt  \textbackslash course} and {\tt  \textbackslash subtitle} are optional, others must be at least
    an empty string

2. Generate the title slide using after beginning of the document by calling
\begin{lstlisting}[language=TeX]
\maketitle
\end{lstlisting}

    \vspace{5pt}

    Hint: Don't use \textbf{ř} in your code snippets, it will break.

\end{frame}

% -----------------------------------------------------------------------------

\begin{frame}[fragile]
    \frametitle{Licence}

    \begin{columns}
        \column{0.25\textwidth}
        \centering

        %LaTeX with PSTricks extensions
%%Creator: inkscape 0.91
%%Please note this file requires PSTricks extensions
\psset{xunit=.5pt,yunit=.5pt,runit=.5pt}
\begin{pspicture}(120,42)
{
\newrgbcolor{curcolor}{0.66666669 0.69803923 0.67058825}
\pscustom[linestyle=none,fillstyle=solid,fillcolor=curcolor]
{
\newpath
\moveto(3.4083297,41.54731536)
\lineto(116.76243,41.34595815)
\curveto(118.34626,41.34595815)(119.76124,41.58079209)(119.76124,38.185433)
\lineto(119.62246,0.856356)
\lineto(0.54733418,0.856356)
\lineto(0.54733418,38.324199)
\curveto(0.54733418,39.9985919)(0.70939995,41.54731536)(3.4083297,41.54731536)
\closepath
}
}
{
\newrgbcolor{curcolor}{0 0 0}
\pscustom[linestyle=none,fillstyle=solid,fillcolor=curcolor]
{
\newpath
\moveto(117.7533,42.00000142)
\lineto(2.2476335,42.00000142)
\curveto(1.0083193,42.00000142)(-0.00002041,40.9917746)(-0.00002041,39.7530859)
\lineto(-0.00002041,0.507022)
\curveto(-0.00002041,0.227065)(0.22707838,0.000002)(0.5075432,0.000002)
\lineto(119.49241,0.000002)
\curveto(119.77289,0.000002)(119.99998,0.227075)(119.99998,0.507022)
\lineto(119.99998,39.7530859)
\curveto(119.99998,40.9917746)(118.99164,42.00000142)(117.7533,42.00000142)
\closepath
\moveto(2.2476335,40.9849779)
\lineto(117.75331,40.9849779)
\curveto(118.43264,40.9849779)(118.98486,40.4323485)(118.98486,39.753076)
\lineto(118.98486,12.532451)
\lineto(36.428267,12.532451)
\curveto(33.402284,7.061445)(27.571583,3.34683)(20.881033,3.34683)
\curveto(14.188536,3.34683)(8.3597829,7.058047)(5.3357378,12.532451)
\lineto(1.0151167,12.532451)
\lineto(1.0151167,39.753076)
\curveto(1.0151068,40.4323485)(1.5682849,40.9849779)(2.2476335,40.9849779)
\closepath
}
}
{
\newrgbcolor{curcolor}{1 1 1}
\pscustom[linestyle=none,fillstyle=solid,fillcolor=curcolor]
{
\newpath
\moveto(34.52218482,22.44874988)
\curveto(34.5269994,14.91907517)(28.42554577,8.81199781)(20.89550614,8.80715775)
\curveto(13.36551857,8.80228299)(7.25722909,14.90308908)(7.25241451,22.43228672)
\lineto(7.25241451,22.44874988)
\curveto(7.24753921,29.97842459)(13.34900152,36.08550194)(20.87904114,36.09036803)
\curveto(28.41003501,36.094705)(34.51827243,29.99446272)(34.52218482,22.46526508)
\lineto(34.52218482,22.44874988)
\closepath
}
}
{
\newrgbcolor{curcolor}{0 0 0}
\pscustom[linestyle=none,fillstyle=solid,fillcolor=curcolor]
{
\newpath
\moveto(31.97172335,33.55331543)
\curveto(34.99481187,30.53008787)(36.5067812,26.82808895)(36.5067812,22.44875855)
\curveto(36.5067812,18.06895108)(35.02096676,14.40624516)(32.04933787,11.46064947)
\curveto(28.89525804,8.35882724)(25.16859873,6.80816767)(20.86735602,6.80816767)
\curveto(16.61852719,6.80816767)(12.95594085,8.34574692)(9.88148812,11.42231928)
\curveto(6.80502281,14.49746043)(5.26774874,18.1727696)(5.26774874,22.44875855)
\curveto(5.26774874,26.72379337)(6.80502281,30.42530654)(9.88148812,33.55331543)
\curveto(12.87831765,36.57749712)(16.54091267,38.08934943)(20.86735602,38.08934943)
\curveto(25.24717616,38.08934943)(28.94768059,36.57749712)(31.97172335,33.55331543)
\closepath
\moveto(11.91658121,31.51940403)
\curveto(9.36027284,28.9377066)(8.08211866,25.91352491)(8.08211866,22.44489864)
\curveto(8.08211866,18.97627238)(9.34766818,15.97776558)(11.87676332,13.44895323)
\curveto(14.40775827,10.91919543)(17.41920505,9.65430785)(20.91394902,9.65430785)
\curveto(24.408693,9.65430785)(27.44629466,10.93181602)(30.02875792,13.48728342)
\curveto(32.48022987,15.86086675)(33.70596151,18.84576411)(33.70596151,22.44489864)
\curveto(33.70596151,26.01638929)(32.45989585,29.04830815)(29.96860602,31.53883368)
\curveto(27.47837452,34.02878672)(24.46016131,35.27428802)(20.91394902,35.27428802)
\curveto(17.36773674,35.27428802)(14.36699482,34.02248076)(11.91658121,31.51940403)
\closepath
\moveto(18.64303688,23.97029967)
\curveto(18.25291701,24.82181276)(17.66869578,25.24783387)(16.88835193,25.24783387)
\curveto(15.51023663,25.24783387)(14.82117899,24.31966039)(14.82117899,22.46426758)
\curveto(14.82117899,20.60844974)(15.51023663,19.6812304)(16.88835193,19.6812304)
\curveto(17.79873284,19.6812304)(18.44892685,20.1329264)(18.83904672,21.03833942)
\lineto(20.74992362,20.02085895)
\curveto(19.83869257,18.40275074)(18.47222769,17.59301139)(16.6506244,17.59301139)
\curveto(15.2452872,17.59301139)(14.11952535,18.02384654)(13.27426707,18.88504844)
\curveto(12.42806323,19.7467274)(12.00490058,20.93499797)(12.00490058,22.44875855)
\curveto(12.00490058,23.93636717)(12.44161346,25.11733427)(13.31313072,25.99208488)
\curveto(14.18454388,26.86688754)(15.27155486,27.30408069)(16.57300121,27.30408069)
\curveto(18.49933681,27.30408069)(19.87745211,26.54521406)(20.71211831,25.02902478)
\lineto(18.64303688,23.97029967)
\closepath
\moveto(27.63362958,23.97029967)
\curveto(27.24255547,24.82181276)(26.66892632,25.24783387)(25.91389589,25.24783387)
\curveto(24.50771722,25.24783387)(23.80404233,24.31966039)(23.80404233,22.46426758)
\curveto(23.80404233,20.60844974)(24.50771722,19.6812304)(25.91389589,19.6812304)
\curveto(26.82618528,19.6812304)(27.46472887,20.1329264)(27.82869385,21.03833942)
\lineto(29.78224269,20.02085895)
\curveto(28.87292012,18.40275074)(27.50846782,17.59301139)(25.68971858,17.59301139)
\curveto(24.28639396,17.59301139)(23.16359026,18.02384654)(22.31823656,18.88504844)
\curveto(21.47488677,19.7467274)(21.0517328,20.93499797)(21.0517328,22.44875855)
\curveto(21.0517328,23.93636717)(21.48071631,25.11733427)(22.33867468,25.99208488)
\curveto(23.1955747,26.86688754)(24.28640264,27.30408069)(25.61210407,27.30408069)
\curveto(27.53463138,27.30408069)(28.91179244,26.54521406)(29.74253758,25.02902478)
\lineto(27.63362958,23.97029967)
\closepath
}
}
{
\newrgbcolor{curcolor}{1 1 1}
\pscustom[linestyle=none,fillstyle=solid,fillcolor=curcolor]
{
\newpath
\moveto(62.50238153,26.77618464)
\curveto(62.50238153,20.84589283)(57.69439216,16.03844146)(51.7634367,16.03844146)
\curveto(45.83248125,16.03844146)(41.02449188,20.84589283)(41.02449188,26.77618464)
\curveto(41.02449188,32.70647644)(45.83248125,37.51392782)(51.7634367,37.51392782)
\curveto(57.69439216,37.51392782)(62.50238153,32.70647644)(62.50238153,26.77618464)
\closepath
}
}
{
\newrgbcolor{curcolor}{0 0 0}
\pscustom[linestyle=none,fillstyle=solid,fillcolor=curcolor]
{
\newpath
\moveto(54.87142746,29.88334337)
\curveto(54.87142746,30.29720675)(54.53563886,30.63248081)(54.12220618,30.63248081)
\lineto(49.37942706,30.63248081)
\curveto(48.96599437,30.63248081)(48.63020578,30.29721669)(48.63020578,29.88334337)
\lineto(48.63020578,25.14060805)
\lineto(49.95297771,25.14060805)
\lineto(49.95297771,19.52452658)
\lineto(53.54767168,19.52452658)
\lineto(53.54767168,25.14060805)
\lineto(54.87141752,25.14060805)
\lineto(54.87141752,29.88334337)
\lineto(54.87142746,29.88334337)
\closepath
}
}
{
\newrgbcolor{curcolor}{0 0 0}
\pscustom[linestyle=none,fillstyle=solid,fillcolor=curcolor]
{
\newpath
\moveto(53.37298867,32.87890741)
\curveto(53.37298867,31.98310883)(52.64671917,31.2569206)(51.75082034,31.2569206)
\curveto(50.85492152,31.2569206)(50.12865202,31.98310883)(50.12865202,32.87890741)
\curveto(50.12865202,33.77470598)(50.85492152,34.50089421)(51.75082034,34.50089421)
\curveto(52.64671917,34.50089421)(53.37298867,33.77470598)(53.37298867,32.87890741)
\closepath
}
}
{
\newrgbcolor{curcolor}{0 0 0}
\pscustom[linestyle=none,fillstyle=solid,fillcolor=curcolor]
{
\newpath
\moveto(51.73578071,38.60754385)
\curveto(48.52442775,38.60754385)(45.80512547,37.48675409)(43.57979187,35.24517459)
\curveto(41.29623266,32.92644658)(40.15493504,30.18173241)(40.15493504,27.01294987)
\curveto(40.15493504,23.84416732)(41.29623266,21.11885952)(43.57979187,18.83798038)
\curveto(45.86335109,16.558552)(48.5826633,15.41787892)(51.73578071,15.41787892)
\curveto(54.92868911,15.41787892)(57.69652764,16.5672963)(60.03831247,18.8675818)
\curveto(62.24423752,21.0509422)(63.34670813,23.76606489)(63.34670813,27.0129598)
\curveto(63.34670813,30.25985471)(62.22481905,33.00360502)(59.98007691,35.24518453)
\curveto(57.73533478,37.48676403)(54.98788863,38.60754385)(51.73578071,38.60754385)
\closepath
\moveto(51.76586246,36.52123515)
\curveto(54.39783128,36.52123515)(56.63189027,35.59306825)(58.46999718,33.73769831)
\curveto(60.32848653,31.90222164)(61.25724425,29.66015524)(61.25724425,27.0129598)
\curveto(61.25724425,24.346358)(60.34789506,22.13340612)(58.52823273,20.37459105)
\curveto(56.61249167,18.48089528)(54.3580403,17.53428587)(51.76586246,17.53428587)
\curveto(49.17174675,17.53428587)(46.93670391,18.47118712)(45.05977993,20.34596343)
\curveto(43.18285595,22.22025284)(42.24438899,24.44242598)(42.24438899,27.0129598)
\curveto(42.24438899,29.58349363)(43.19255525,31.82459617)(45.08888777,33.73769831)
\curveto(46.90856998,35.59305832)(49.13389364,36.52123515)(51.76586246,36.52123515)
\closepath
}
}
{
\newrgbcolor{curcolor}{1 1 1}
\pscustom[linestyle=none,fillstyle=solid,fillcolor=curcolor]
{
\newpath
\moveto(48.40117905,9.25256646)
\curveto(48.7165852,9.25256646)(49.00482139,9.22442574)(49.2649137,9.16911809)
\curveto(49.52500601,9.11381044)(49.74821911,9.02259157)(49.93455301,8.89547142)
\curveto(50.11991301,8.76932506)(50.26354415,8.60047078)(50.36642034,8.39087606)
\curveto(50.46832262,8.18030753)(50.51976071,7.92120818)(50.51976071,7.61166023)
\curveto(50.51976071,7.27784686)(50.44406443,7.00032488)(50.29169796,6.77810061)
\curveto(50.14030539,6.55491248)(49.91514447,6.37344854)(49.6181829,6.23080727)
\curveto(50.0277299,6.11339527)(50.33343675,5.90766592)(50.53529351,5.61364902)
\curveto(50.73715026,5.31963212)(50.83808858,4.96543859)(50.83808858,4.55107837)
\curveto(50.83808858,4.217265)(50.7730655,3.92809721)(50.64301935,3.68355512)
\curveto(50.51297319,3.43998683)(50.3373125,3.24009032)(50.11701117,3.08580324)
\curveto(49.89768375,2.93054236)(49.64632677,2.81604182)(49.36487811,2.74229166)
\curveto(49.08246548,2.6675677)(48.79325539,2.63070256)(48.49531991,2.63070256)
\lineto(45.27911729,2.63070256)
\lineto(45.27911729,9.2525764)
\lineto(48.40117905,9.2525764)
\lineto(48.40117905,9.25256646)
\closepath
\moveto(48.21484515,6.57431885)
\curveto(48.47493746,6.57431885)(48.68844133,6.6364232)(48.85536669,6.7596581)
\curveto(49.02326596,6.88289301)(49.10672367,7.08279946)(49.10672367,7.36032144)
\curveto(49.10672367,7.51460852)(49.0785798,7.64172867)(49.02326596,7.73974424)
\curveto(48.96697821,7.83872367)(48.8932198,7.91538528)(48.80005285,7.97166672)
\curveto(48.7068859,8.02697437)(48.60013397,8.06579705)(48.47978711,8.08714107)
\curveto(48.35847629,8.10848509)(48.23327979,8.11916704)(48.10323363,8.11916704)
\lineto(46.7387229,8.11916704)
\lineto(46.7387229,6.57431885)
\lineto(48.21484515,6.57431885)
\closepath
\moveto(48.30025067,3.76410198)
\curveto(48.44290791,3.76410198)(48.57877762,3.77768544)(48.70882377,3.80582617)
\curveto(48.83886993,3.83396689)(48.95338329,3.88055012)(49.0533477,3.94459213)
\curveto(49.15233821,4.00960793)(49.23094627,4.09791535)(49.29014579,4.20950444)
\curveto(49.34837141,4.32012968)(49.37846309,4.46277095)(49.37846309,4.63550054)
\curveto(49.37846309,4.97513681)(49.28238436,5.2177313)(49.09022691,5.36328402)
\curveto(48.89806945,5.50787289)(48.6438007,5.58064925)(48.32839455,5.58064925)
\lineto(46.73873284,5.58064925)
\lineto(46.73873284,3.76410198)
\lineto(48.30025067,3.76410198)
\closepath
}
}
{
\newrgbcolor{curcolor}{1 1 1}
\pscustom[linestyle=none,fillstyle=solid,fillcolor=curcolor]
{
\newpath
\moveto(51.09525917,9.25256646)
\lineto(52.72762364,9.25256646)
\lineto(54.27846827,6.63738706)
\lineto(55.82057756,9.25256646)
\lineto(57.4442067,9.25256646)
\lineto(54.98498679,5.17211193)
\lineto(54.98498679,2.63068269)
\lineto(53.52537124,2.63068269)
\lineto(53.52537124,5.20898701)
\lineto(51.09525917,9.25256646)
\closepath
}
}
{
\newrgbcolor{curcolor}{1 1 1}
\pscustom[linestyle=none,fillstyle=solid,fillcolor=curcolor]
{
\newpath
\moveto(98.44056654,4.26772354)
\curveto(98.52111247,4.11343647)(98.62690044,3.98825397)(98.75985837,3.89218599)
\curveto(98.8928163,3.79611802)(99.04809455,3.72527931)(99.22763099,3.67870602)
\curveto(99.40620346,3.63212278)(99.59058955,3.60884111)(99.78274701,3.60884111)
\curveto(99.91181926,3.60884111)(100.05060075,3.61951312)(100.19909147,3.64183094)
\curveto(100.3466083,3.66317496)(100.48538979,3.70490908)(100.61543594,3.76701343)
\curveto(100.74450819,3.82814399)(100.85223403,3.91353995)(100.93957742,4.02222753)
\curveto(101.02498295,4.12994132)(101.06864967,4.26773348)(101.06864967,4.43464016)
\curveto(101.06864967,4.61415651)(101.01138803,4.75874538)(100.89687466,4.87034441)
\curveto(100.78332527,4.98193351)(100.63290661,5.07412617)(100.44851058,5.14884019)
\curveto(100.26315059,5.22259035)(100.05254856,5.28760615)(99.81866225,5.34388759)
\curveto(99.58380203,5.39919524)(99.34603004,5.46129959)(99.1053562,5.52922685)
\curveto(98.85788491,5.5913312)(98.61720114,5.66701901)(98.38234092,5.75630023)
\curveto(98.14845461,5.8455715)(97.93785258,5.96201964)(97.75249259,6.10467085)
\curveto(97.56809656,6.24634826)(97.4176779,6.42392696)(97.3041285,6.63740693)
\curveto(97.18961514,6.85088691)(97.13235349,7.10901246)(97.13235349,7.41177365)
\curveto(97.13235349,7.75237378)(97.205138,8.04737441)(97.34974304,8.29772947)
\curveto(97.49531205,8.54808452)(97.68553164,8.75671539)(97.91941795,8.92459587)
\curveto(98.15427816,9.09150255)(98.42019403,9.21473746)(98.71715561,9.29528432)
\curveto(99.01315322,9.37582124)(99.31012474,9.41560777)(99.60612235,9.41560777)
\curveto(99.95258415,9.41560777)(100.284497,9.37679503)(100.60281492,9.29915963)
\curveto(100.92016889,9.22249802)(101.20354548,9.09635166)(101.45005281,8.92362208)
\curveto(101.6975241,8.74992863)(101.89356724,8.52867816)(102.03913625,8.25988059)
\curveto(102.1837413,7.99011917)(102.2565258,7.66406636)(102.2565258,7.28076825)
\lineto(100.84348876,7.28076825)
\curveto(100.83086774,7.47872711)(100.79011279,7.64272234)(100.71926616,7.77178015)
\curveto(100.64841953,7.90181175)(100.55427867,8.00370263)(100.43685353,8.07841665)
\curveto(100.32039236,8.15216681)(100.18647046,8.2055368)(100.03604186,8.23561518)
\curveto(99.88561326,8.26666736)(99.72063571,8.28219841)(99.54206324,8.28219841)
\curveto(99.42560206,8.28219841)(99.30817693,8.26957881)(99.19171575,8.24532333)
\curveto(99.07429061,8.22009406)(98.96850265,8.17739608)(98.87339783,8.11529173)
\curveto(98.7773191,8.05318738)(98.69968494,7.97652577)(98.63757364,7.88336924)
\curveto(98.57643625,7.79021272)(98.5453806,7.67280072)(98.5453806,7.53112331)
\curveto(98.5453806,7.4010917)(98.56963879,7.29532551)(98.61913901,7.21575245)
\curveto(98.66862929,7.13521553)(98.76568192,7.06049158)(98.91126087,6.99256432)
\curveto(99.05683982,6.92463707)(99.25675871,6.85670981)(99.51296533,6.78878256)
\curveto(99.76917196,6.72085531)(100.10496056,6.63352168)(100.51839325,6.52871935)
\curveto(100.64164194,6.50446388)(100.81342689,6.45885444)(101.03178041,6.39383864)
\curveto(101.25110783,6.32882284)(101.46947128,6.22595816)(101.68492296,6.08331689)
\curveto(101.90133861,5.94067562)(102.08767251,5.75047732)(102.24586253,5.51273193)
\curveto(102.40307866,5.27498654)(102.48168672,4.9702877)(102.48168672,4.59863539)
\curveto(102.48168672,4.2958742)(102.4224872,4.01446697)(102.30506206,3.75440377)
\curveto(102.18763692,3.49434056)(102.01294019,3.27018857)(101.78100169,3.08192793)
\curveto(101.54905325,2.89270343)(101.26179097,2.74617691)(100.91824095,2.64137457)
\curveto(100.5756549,2.53559844)(100.17872891,2.48320225)(99.72647914,2.48320225)
\curveto(99.36157277,2.48320225)(99.00831351,2.52881168)(98.66476348,2.61808296)
\curveto(98.32121346,2.70832803)(98.01841839,2.84903165)(97.75638821,3.0411676)
\curveto(97.49338412,3.23330355)(97.2847299,3.4788095)(97.12945165,3.77574779)
\curveto(96.97514731,4.07268608)(96.90041499,4.42493201)(96.90721245,4.83345939)
\lineto(98.32024949,4.83345939)
\curveto(98.32021968,4.61123512)(98.36001066,4.42201062)(98.44056654,4.26772354)
\closepath
}
}
{
\newrgbcolor{curcolor}{1 1 1}
\pscustom[linestyle=none,fillstyle=solid,fillcolor=curcolor]
{
\newpath
\moveto(106.64604566,9.25256646)
\lineto(109.12273623,2.63069262)
\lineto(107.61071862,2.63069262)
\lineto(107.10994254,4.10567591)
\lineto(104.63325196,4.10567591)
\lineto(104.11306734,2.63069262)
\lineto(102.64762824,2.63069262)
\lineto(105.15246268,9.25256646)
\lineto(106.64604566,9.25256646)
\closepath
\moveto(106.72951331,5.19250203)
\lineto(105.89489644,7.62039458)
\lineto(105.8754879,7.62039458)
\lineto(105.01369113,5.19250203)
\lineto(106.72951331,5.19250203)
\closepath
}
}
{
\newrgbcolor{curcolor}{1 1 1}
\pscustom[linestyle=none,fillstyle=solid,fillcolor=curcolor]
{
\newpath
\moveto(72.51783412,9.25256646)
\lineto(75.28469875,4.81016778)
\lineto(75.3002216,4.81016778)
\lineto(75.3002216,9.25256646)
\lineto(76.6666702,9.25256646)
\lineto(76.6666702,2.63069262)
\lineto(75.20996643,2.63069262)
\lineto(72.45377501,7.06435695)
\lineto(72.43534037,7.06435695)
\lineto(72.43534037,2.63069262)
\lineto(71.06889177,2.63069262)
\lineto(71.06889177,9.25256646)
\lineto(72.51783412,9.25256646)
\closepath
}
}
{
\newrgbcolor{curcolor}{1 1 1}
\pscustom[linestyle=none,fillstyle=solid,fillcolor=curcolor]
{
\newpath
\moveto(82.19554981,7.48841539)
\curveto(82.10918032,7.62814521)(82.00048057,7.75041625)(81.87043442,7.85521858)
\curveto(81.74038826,7.96002092)(81.59384535,8.04250543)(81.42983176,8.10072453)
\curveto(81.26581818,8.15991743)(81.09404317,8.18903195)(80.9154707,8.18903195)
\curveto(80.58744353,8.18903195)(80.30891659,8.1259538)(80.07987993,7.99883365)
\curveto(79.85084326,7.8726873)(79.66548327,7.70286916)(79.52379,7.48937925)
\curveto(79.38113277,7.27589927)(79.27728268,7.03330477)(79.2122596,6.76159576)
\curveto(79.14723652,6.48988674)(79.11520696,6.20847951)(79.11520696,5.91833793)
\curveto(79.11520696,5.63984215)(79.14723652,5.36910693)(79.2122596,5.10709613)
\curveto(79.27728268,4.84412147)(79.38112283,4.60734988)(79.52379,4.39774521)
\curveto(79.66548327,4.18717669)(79.85084326,4.01929621)(80.07987993,3.89217606)
\curveto(80.30891659,3.7650559)(80.58744353,3.70197776)(80.9154707,3.70197776)
\curveto(81.35995904,3.70197776)(81.70835871,3.83783227)(81.95874178,4.11050514)
\curveto(82.20912485,4.38221416)(82.36246523,4.74125679)(82.41777907,5.18665925)
\lineto(83.82790434,5.18665925)
\curveto(83.79102513,4.77230897)(83.6949464,4.39773527)(83.54064206,4.06393185)
\curveto(83.38633771,3.72915462)(83.18155924,3.444826)(82.92826439,3.20902821)
\curveto(82.67496954,2.97323041)(82.37799803,2.79370412)(82.03735978,2.66949542)
\curveto(81.6976855,2.54528672)(81.32307983,2.48318237)(80.9154707,2.48318237)
\curveto(80.40790709,2.48318237)(79.95177164,2.57148979)(79.54611032,2.74809469)
\curveto(79.14141296,2.92373573)(78.79883685,3.16730402)(78.52029996,3.47587818)
\curveto(78.24177302,3.78542613)(78.02729525,4.14931787)(77.87880452,4.56658955)
\curveto(77.73032373,4.98482508)(77.65559141,5.43411279)(77.65559141,5.91736413)
\curveto(77.65559141,6.41226127)(77.73032373,6.8712472)(77.87880452,7.29530564)
\curveto(78.02728531,7.71936408)(78.24176308,8.08907872)(78.52029996,8.40444958)
\curveto(78.79883685,8.71982044)(79.14141296,8.96727398)(79.54611032,9.14679033)
\curveto(79.95177164,9.32630668)(80.40790709,9.41558789)(80.9154707,9.41558789)
\curveto(81.28037707,9.41558789)(81.624901,9.3631917)(81.94904248,9.25741557)
\curveto(82.27415787,9.15261323)(82.56530583,8.99929002)(82.82151246,8.79745585)
\curveto(83.07869299,8.59658554)(83.29025899,8.34720428)(83.45718435,8.05026599)
\curveto(83.62410971,7.7533277)(83.72892377,7.41272757)(83.7725905,7.02942946)
\lineto(82.36246523,7.02942946)
\curveto(82.33724307,7.19633615)(82.2819193,7.34964943)(82.19554981,7.48841539)
\closepath
}
}
{
\newrgbcolor{curcolor}{1 1 1}
\pscustom[linestyle=none,fillstyle=solid,fillcolor=curcolor]
{
\newpath
\moveto(114.57975107,27.01925344)
\curveto(114.58460193,21.17416329)(109.84857889,16.43191951)(104.00241227,16.42755982)
\curveto(98.15617111,16.42415032)(93.41438419,21.15816534)(93.4094588,27.0046963)
\lineto(93.4094588,27.01925344)
\curveto(93.4055893,32.8643436)(98.14161854,37.60658737)(103.98687214,37.61094706)
\curveto(109.8331133,37.61530675)(114.57490022,32.87985093)(114.57975107,27.0352514)
\lineto(114.57975107,27.01925344)
\closepath
}
}
{
\newrgbcolor{curcolor}{0 0 0}
\pscustom[linestyle=none,fillstyle=solid,fillcolor=curcolor]
{
\newpath
\moveto(103.91889187,38.61434034)
\curveto(100.70659583,38.61434034)(97.98727942,37.49306775)(95.76290534,35.25245399)
\curveto(93.47939556,32.93322298)(92.33710963,30.18851228)(92.33710963,27.01925965)
\curveto(92.33710963,23.85000703)(93.47939556,21.12564776)(95.76290534,18.84477829)
\curveto(98.04550829,16.56436218)(100.76482471,15.42417897)(103.91889187,15.42417897)
\curveto(107.11185282,15.42417897)(109.87870884,16.57456585)(112.22044749,18.87340196)
\curveto(114.42639951,21.05772478)(115.52986619,23.77287405)(115.52986619,27.01926586)
\curveto(115.52986619,30.26569494)(114.40797744,33.01040564)(112.16221862,35.2524602)
\curveto(109.91752809,37.49306775)(107.17007549,38.61434034)(103.91889187,38.61434034)
\closepath
\moveto(103.94800321,36.52705478)
\curveto(106.57997941,36.52705478)(108.8140552,35.59985672)(110.65311872,33.74398253)
\curveto(112.51067884,31.90854051)(113.43938436,29.66792675)(113.43938436,27.01925965)
\curveto(113.43938436,24.35266938)(112.53008846,22.14018247)(110.71036003,20.38039535)
\curveto(108.79563313,18.48718433)(106.5401726,17.54057881)(103.94800321,17.54057881)
\curveto(101.35484625,17.54057881)(99.11985743,18.47797433)(97.24288769,20.35226852)
\curveto(95.36599248,22.22702228)(94.4275045,24.4487192)(94.4275045,27.01925965)
\curveto(94.4275045,29.58980011)(95.37470663,31.83135784)(97.27199902,33.74398253)
\curveto(99.0907461,35.59985672)(101.316027,36.52705478)(103.94800321,36.52705478)
\closepath
}
}
{
\newrgbcolor{curcolor}{0 0 0}
\pscustom[linestyle=none,fillstyle=solid,fillcolor=curcolor]
{
\newpath
\moveto(98.77920454,28.65532348)
\curveto(99.24116604,31.57326666)(101.294723,33.13264465)(103.86945168,33.13264465)
\curveto(107.57085164,33.13264465)(109.82721899,30.44664693)(109.82721899,26.86498748)
\curveto(109.82721899,23.36967723)(107.4262639,20.6555092)(103.81114206,20.6555092)
\curveto(101.32382812,20.6555092)(99.09657209,22.18532575)(98.69185191,25.18961192)
\lineto(101.61299737,25.18961192)
\curveto(101.70033757,23.63019046)(102.71260076,23.08098748)(104.15861472,23.08098748)
\curveto(105.80653846,23.08098748)(106.87694978,24.61174802)(106.87694978,26.95133667)
\curveto(106.87694978,29.40736384)(105.95211376,30.70622239)(104.21586224,30.70622239)
\curveto(102.94354735,30.70622239)(101.84493152,30.24385302)(101.61300358,28.65532348)
\lineto(102.46217623,28.66116124)
\lineto(100.16312012,26.36133768)
\lineto(97.86398948,28.66116124)
\lineto(98.77920454,28.65532348)
\closepath
}
}
{
\newrgbcolor{curcolor}{1 1 1}
\pscustom[linestyle=none,fillstyle=solid,fillcolor=curcolor]
{
\newpath
\moveto(87.80189623,27.00915545)
\curveto(87.80672823,21.50856845)(83.35029323,17.04723845)(77.84864523,17.04285145)
\curveto(72.34793123,17.03947845)(67.88566023,21.49496145)(67.88179723,26.99461445)
\lineto(67.88179723,27.00915545)
\curveto(67.87789923,32.51023195)(72.33436923,36.97156095)(77.83504823,36.97593685)
\curveto(83.33572823,36.97934415)(87.79799823,32.52382695)(87.80189623,27.02420845)
\lineto(87.80189623,27.00915545)
\closepath
}
}
{
\newrgbcolor{curcolor}{0 0 0}
\pscustom[linestyle=none,fillstyle=solid,fillcolor=curcolor]
{
\newpath
\moveto(86.07086623,35.24481185)
\curveto(83.82574623,37.48710385)(81.07808623,38.60754385)(77.82718623,38.60754385)
\curveto(74.61450623,38.60754385)(71.89586623,37.48710385)(69.67056623,35.24481185)
\curveto(67.38651623,32.92536985)(66.24554623,30.18124945)(66.24554623,27.01104145)
\curveto(66.24554623,23.84153945)(67.38651623,21.11582145)(69.67056623,18.83601845)
\curveto(71.95390623,16.55479845)(74.67254623,15.41453945)(77.82718623,15.41453945)
\curveto(81.02004623,15.41453945)(83.78751623,16.56470545)(86.12889623,18.86432845)
\curveto(88.33509623,21.04858045)(89.43854623,23.76438645)(89.43854623,27.01104145)
\curveto(89.43854623,30.25840245)(88.31598623,33.00322985)(86.07086623,35.24481185)
\closepath
\moveto(84.61917623,20.37193145)
\curveto(82.70317623,18.47787545)(80.44885623,17.53155345)(77.85620623,17.53155345)
\curveto(75.26284623,17.53155345)(73.02762623,18.46867345)(71.15055623,20.34362145)
\curveto(69.27348623,22.21856945)(68.33495623,24.44104245)(68.33495623,27.01104445)
\curveto(68.33495623,28.09821645)(68.50659623,29.12593545)(68.84598623,30.09490645)
\lineto(71.91886623,28.73487945)
\lineto(71.69732623,28.73487945)
\lineto(71.69732623,27.35680545)
\lineto(72.78485623,27.35680545)
\curveto(72.78485623,27.16216145)(72.76539623,26.96822445)(72.76539623,26.77429145)
\lineto(72.76539623,26.44375145)
\lineto(71.69732623,26.44375145)
\lineto(71.69732623,25.06567745)
\lineto(72.95932623,25.06567745)
\curveto(73.13414623,24.03654545)(73.52237623,23.20205445)(74.04649623,22.54168445)
\curveto(75.13366623,21.10486345)(76.88085623,20.32841245)(78.78341623,20.32841245)
\curveto(80.02629623,20.32841245)(81.15169623,20.69717345)(81.81206623,21.06664145)
\lineto(81.34562623,23.22116645)
\curveto(80.93793623,23.00741445)(80.02629623,22.71650845)(79.13305623,22.71650845)
\curveto(78.16196623,22.71650845)(77.24997623,23.00741045)(76.62853623,23.70671245)
\curveto(76.33727623,24.03654545)(76.12387623,24.48316345)(75.98833623,25.06567745)
\lineto(80.20890623,25.06567745)
\lineto(86.20604623,22.41109345)
\curveto(85.79057623,21.67497945)(85.26327623,20.99479045)(84.61917623,20.37193145)
\closepath
\moveto(77.07692623,26.44374445)
\lineto(77.05144623,26.46320845)
\lineto(77.09532623,26.44374445)
\lineto(77.07692623,26.44374445)
\closepath
\moveto(80.68737623,27.35679845)
\lineto(80.86079623,27.35679845)
\lineto(80.86079623,28.73487245)
\lineto(77.57379623,28.73487245)
\lineto(76.23854623,29.32588045)
\curveto(76.35355623,29.58422545)(76.48945623,29.81532145)(76.64799623,29.99616145)
\curveto(77.24997623,30.73509845)(78.10393623,31.04511245)(79.03609623,31.04511245)
\curveto(79.89040623,31.04511245)(80.68596623,30.79313845)(81.19062623,30.57938245)
\lineto(81.73420623,32.79265485)
\curveto(81.03490623,33.10266985)(80.00647623,33.37446285)(78.82233623,33.37446285)
\curveto(76.99764623,33.37446285)(75.44439623,32.63693985)(74.33775623,31.39476145)
\curveto(74.09215623,31.11093745)(73.87910623,30.79242845)(73.68587623,30.45551745)
\lineto(69.87227623,32.14325485)
\curveto(70.24174623,32.69745885)(70.67633623,33.22971985)(71.17957623,33.73720785)
\curveto(72.99931623,35.59304385)(75.22391623,36.52096285)(77.85619623,36.52096285)
\curveto(80.48777623,36.52096285)(82.72299623,35.59304385)(84.56184623,33.73720785)
\curveto(86.41909623,31.90118845)(87.34772623,29.65960645)(87.34772623,27.01103745)
\curveto(87.34772623,26.13832845)(87.24862623,25.31516145)(87.05327623,24.53942045)
\lineto(80.68737623,27.35679845)
\closepath
}
}
\end{pspicture}
 \\ {\tt cc-by-nc-sa}

        \vspace{3pt}%LaTeX with PSTricks extensions
%%Creator: inkscape 0.91
%%Please note this file requires PSTricks extensions
\psset{xunit=.5pt,yunit=.5pt,runit=.5pt}
\begin{pspicture}(120,42)
{
\newrgbcolor{curcolor}{0.66666669 0.69803923 0.67058825}
\pscustom[linestyle=none,fillstyle=solid,fillcolor=curcolor]
{
\newpath
\moveto(3.40930731,41.55411064)
\lineto(116.76249121,41.35227647)
\curveto(118.34631661,41.35227647)(119.76128014,41.58759731)(119.76128014,38.19222828)
\lineto(119.62249977,0.86315172)
\lineto(0.54832486,0.86315172)
\lineto(0.54832486,38.33099424)
\curveto(0.54832486,40.00490024)(0.71039927,41.55411064)(3.40930731,41.55411064)
\closepath
}
}
{
\newrgbcolor{curcolor}{0 0 0}
\pscustom[linestyle=none,fillstyle=solid,fillcolor=curcolor]
{
\newpath
\moveto(117.75335226,42)
\lineto(2.24764654,42)
\curveto(1.00834229,42)(0.00001072,40.99177314)(0.00001072,39.75308449)
\lineto(0.00001072,0.50702054)
\curveto(0.00001072,0.22706407)(0.22710768,0.00000063)(0.50757024,0.00000063)
\lineto(119.49245466,0.00000063)
\curveto(119.77292715,0.00000063)(120.00001418,0.227074)(120.00001418,0.50702054)
\lineto(120.00001418,39.75308449)
\curveto(120.00001418,40.99177314)(118.99168261,42)(117.75335226,42)
\closepath
\moveto(2.24764654,40.98498638)
\lineto(117.75335226,40.98498638)
\curveto(118.43268541,40.98498638)(118.98489514,40.43235703)(118.98489514,39.75308449)
\lineto(118.98489514,12.53875959)
\lineto(36.42896882,12.53875959)
\curveto(33.40301014,7.06824078)(27.57235631,3.35362559)(20.88186043,3.35362559)
\curveto(14.18941677,3.35362559)(8.36071073,7.06484243)(5.33668991,12.53875959)
\lineto(1.0151397,12.53875959)
\lineto(1.0151397,39.75308449)
\curveto(1.01512976,40.43234709)(1.56830345,40.98498638)(2.24764654,40.98498638)
\closepath
}
}
{
\newrgbcolor{curcolor}{1 1 1}
\pscustom[linestyle=none,fillstyle=solid,fillcolor=curcolor]
{
\newpath
\moveto(34.52290709,22.45507845)
\curveto(34.52778236,14.92588115)(28.42631691,8.81880407)(20.89638969,8.81346092)
\curveto(13.36641042,8.80912394)(7.25822196,14.90936595)(7.25335537,22.43856325)
\lineto(7.25335537,22.45507845)
\curveto(7.2484801,29.98522989)(13.3498935,36.09230697)(20.87987277,36.09669599)
\curveto(28.41080627,36.10103296)(34.51904678,30.00073891)(34.52290709,22.47154161)
\lineto(34.52290709,22.45507845)
\closepath
}
}
{
\newrgbcolor{curcolor}{0 0 0}
\pscustom[linestyle=none,fillstyle=solid,fillcolor=curcolor]
{
\newpath
\moveto(31.97251811,33.55963483)
\curveto(34.99553035,30.53640741)(36.50753961,26.83440866)(36.50753961,22.45507845)
\curveto(36.50753961,18.07527118)(35.02173708,14.41304249)(32.05013201,11.46694384)
\curveto(28.89607746,8.36514778)(25.16939597,6.81448828)(20.8682398,6.81448828)
\curveto(16.61949707,6.81448828)(12.95688803,8.35201542)(9.8824079,11.42861366)
\curveto(6.80596725,14.50425776)(5.26875755,18.17956676)(5.26875755,22.45507845)
\curveto(5.26875755,26.73006103)(6.80596725,30.43157404)(9.8824079,33.55963483)
\curveto(12.87926546,36.58429345)(16.54183113,38.09566862)(20.8682398,38.09566862)
\curveto(25.24802483,38.09566862)(28.94849959,36.58429345)(31.97251811,33.55963483)
\closepath
\moveto(11.91748468,31.52620059)
\curveto(9.36124885,28.94450327)(8.08311359,25.92032172)(8.08311359,22.4511665)
\curveto(8.08311359,18.9830695)(9.34865297,15.98456284)(11.87772784,13.45528222)
\curveto(14.40875455,10.92600159)(17.42012514,9.66111407)(20.9148411,9.66111407)
\curveto(24.40955706,9.66111407)(27.44713437,10.93859616)(30.02957693,13.49408947)
\curveto(32.48102923,15.86714357)(33.70675104,18.8520408)(33.70675104,22.45117517)
\curveto(33.70675104,26.02314273)(32.46064333,29.05511349)(29.96942551,31.54558687)
\curveto(27.47916192,34.03559185)(24.4609729,35.28109309)(20.9148411,35.28109309)
\curveto(17.36870929,35.28109309)(14.36798275,34.02927721)(11.91748468,31.52620059)
\closepath
\moveto(18.64387776,23.97709657)
\curveto(18.25376101,24.82860962)(17.66954447,25.25415364)(16.8892676,25.25415364)
\curveto(15.5112154,25.25415364)(14.82216327,24.32645727)(14.82216327,22.47106454)
\curveto(14.82216327,20.61524679)(15.5112154,19.68802749)(16.8892676,19.68802749)
\curveto(17.79958916,19.68802749)(18.44983001,20.13972347)(18.83994675,21.04507574)
\lineto(20.75080833,20.02764735)
\curveto(19.83953253,18.40953921)(18.47307861,17.59974786)(16.65148992,17.59974786)
\curveto(15.24627676,17.59974786)(14.12051526,18.0306437)(13.27521171,18.89183689)
\curveto(12.4289626,19.75351582)(12.00580334,20.94125722)(12.00580334,22.45506978)
\curveto(12.00580334,23.9431554)(12.44251272,25.1240704)(13.31402299,25.99887302)
\curveto(14.18553327,26.87367563)(15.27248348,27.31033965)(16.57386735,27.31033965)
\curveto(18.50029162,27.31033965)(19.8783525,26.55200217)(20.71295128,25.03576091)
\lineto(18.64387776,23.97709657)
\closepath
\moveto(27.63445911,23.97709657)
\curveto(27.24333609,24.82860962)(26.66976358,25.25415364)(25.91473921,25.25415364)
\curveto(24.50851976,25.25415364)(23.80490256,24.32645727)(23.80490256,22.47106454)
\curveto(23.80490256,20.61524679)(24.50851976,19.68802749)(25.91473921,19.68802749)
\curveto(26.82702129,19.68802749)(27.46555976,20.13972347)(27.82952183,21.04507574)
\lineto(29.78310705,20.02764735)
\curveto(28.87373972,18.40953921)(27.50923763,17.59974786)(25.6905637,17.59974786)
\curveto(24.28725033,17.59974786)(23.16440358,18.0306437)(22.3191087,18.89183689)
\curveto(21.47576567,19.75351582)(21.05266714,20.94125722)(21.05266714,22.45506978)
\curveto(21.05266714,23.9431554)(21.48159517,25.1240704)(22.3394946,25.99887302)
\curveto(23.19643981,26.87367563)(24.287259,27.31033965)(25.6129498,27.31033965)
\curveto(27.53546171,27.31033965)(28.91255968,26.55200217)(29.74329816,25.03576091)
\lineto(27.63445911,23.97709657)
\closepath
}
}
{
\newrgbcolor{curcolor}{1 1 1}
\pscustom[linestyle=none,fillstyle=solid,fillcolor=curcolor]
{
\newpath
\moveto(48.0931783,9.25257498)
\curveto(48.40858191,9.25257498)(48.69778967,9.22443426)(48.95787989,9.16912661)
\curveto(49.21797011,9.11381896)(49.44020752,9.02260009)(49.62653992,8.89547993)
\curveto(49.81189842,8.76933358)(49.95552841,8.6004793)(50.05840377,8.39088458)
\curveto(50.16030523,8.18031605)(50.21174291,7.9212167)(50.21174291,7.61166875)
\curveto(50.21174291,7.27785538)(50.13604723,7.0003334)(49.98368199,6.77810913)
\curveto(49.83229064,6.554921)(49.60810543,6.37345706)(49.31017235,6.23081579)
\curveto(49.71971606,6.11340379)(50.02542045,5.90767443)(50.22727558,5.61365754)
\curveto(50.42913071,5.31964064)(50.53006821,4.96544711)(50.53006821,4.55108689)
\curveto(50.53006821,4.21727352)(50.46504566,3.92810573)(50.33500055,3.68356364)
\curveto(50.20495544,3.43999535)(50.02929616,3.24009884)(49.8099705,3.08581176)
\curveto(49.58967095,2.93055088)(49.338316,2.81605033)(49.05687954,2.74230018)
\curveto(48.77446918,2.66757622)(48.48526142,2.63071108)(48.18732834,2.63071108)
\lineto(44.97115161,2.63071108)
\lineto(44.97115161,9.25258492)
\lineto(48.0931783,9.25258492)
\lineto(48.0931783,9.25257498)
\closepath
\moveto(47.90780986,6.57431743)
\curveto(48.16692618,6.57431743)(48.38140223,6.63642178)(48.54832625,6.75965669)
\curveto(48.71525026,6.88289159)(48.7987073,7.08279804)(48.7987073,7.36032002)
\curveto(48.7987073,7.5146071)(48.77056366,7.64172725)(48.71525026,7.73974282)
\curveto(48.65896297,7.83872225)(48.58520515,7.91538386)(48.49203895,7.97166531)
\curveto(48.39887275,8.02697296)(48.29212168,8.06579563)(48.17177579,8.08713965)
\curveto(48.05142991,8.10848367)(47.92527045,8.11916562)(47.79522534,8.11916562)
\lineto(46.4307256,8.11916562)
\lineto(46.4307256,6.57431743)
\lineto(47.90780986,6.57431743)
\closepath
\moveto(47.99225074,3.76410056)
\curveto(48.13490682,3.76410056)(48.27077544,3.77768403)(48.40082055,3.80582475)
\curveto(48.53086566,3.83396547)(48.6453781,3.8805487)(48.7453417,3.94459071)
\curveto(48.84433141,4.00960651)(48.92293884,4.09791393)(48.98213789,4.20950302)
\curveto(49.04133693,4.32012826)(49.07045447,4.46276953)(49.07045447,4.63549912)
\curveto(49.07045447,4.97513539)(48.97437652,5.21772988)(48.78222061,5.36328261)
\curveto(48.5900647,5.50787147)(48.33677189,5.58064783)(48.02039438,5.58064783)
\lineto(46.43074547,5.58064783)
\lineto(46.43074547,3.76410056)
\lineto(47.99225074,3.76410056)
\closepath
}
}
{
\newrgbcolor{curcolor}{1 1 1}
\pscustom[linestyle=none,fillstyle=solid,fillcolor=curcolor]
{
\newpath
\moveto(50.78723674,9.25257498)
\lineto(52.42055202,9.25257498)
\lineto(53.97138417,6.63739558)
\lineto(55.51250715,9.25257498)
\lineto(57.13612321,9.25257498)
\lineto(54.677897,5.17212045)
\lineto(54.677897,2.63069121)
\lineto(53.2182932,2.63069121)
\lineto(53.2182932,5.20899553)
\lineto(50.78723674,9.25257498)
\closepath
}
}
{
\newrgbcolor{curcolor}{1 1 1}
\pscustom[linestyle=none,fillstyle=solid,fillcolor=curcolor]
{
\newpath
\moveto(72.51728147,9.25257498)
\lineto(75.28412383,4.8101763)
\lineto(75.29964656,4.8101763)
\lineto(75.29964656,9.25257498)
\lineto(76.66608416,9.25257498)
\lineto(76.66608416,2.63070114)
\lineto(75.20939211,2.63070114)
\lineto(72.45322288,7.06436547)
\lineto(72.43478839,7.06436547)
\lineto(72.43478839,2.63070114)
\lineto(71.06835079,2.63070114)
\lineto(71.06835079,9.25257498)
\lineto(72.51728147,9.25257498)
\closepath
}
}
{
\newrgbcolor{curcolor}{1 1 1}
\pscustom[linestyle=none,fillstyle=solid,fillcolor=curcolor]
{
\newpath
\moveto(82.19491926,7.48841397)
\curveto(82.10855046,7.62814379)(81.99985159,7.75041483)(81.86980649,7.85521717)
\curveto(81.73976138,7.9600195)(81.59321964,8.04250401)(81.42920738,8.10072312)
\curveto(81.26519511,8.15991601)(81.09342149,8.18903053)(80.91485045,8.18903053)
\curveto(80.58682593,8.18903053)(80.30830122,8.12595239)(80.07926641,7.99883223)
\curveto(79.85023159,7.87268588)(79.66487309,7.70286774)(79.52318096,7.48937783)
\curveto(79.38052487,7.27589785)(79.27667562,7.03330336)(79.21165306,6.76159434)
\curveto(79.14663051,6.48988533)(79.11460121,6.2084781)(79.11460121,5.91833651)
\curveto(79.11460121,5.63984073)(79.14663051,5.36910551)(79.21165306,5.10709471)
\curveto(79.27667562,4.84412005)(79.38051494,4.60734846)(79.52318096,4.39774379)
\curveto(79.66487309,4.18717527)(79.85023159,4.01929479)(80.07926641,3.89217464)
\curveto(80.30830122,3.76505449)(80.58682593,3.70197634)(80.91485045,3.70197634)
\curveto(81.35933521,3.70197634)(81.70773208,3.83783085)(81.95811314,4.11050372)
\curveto(82.20849419,4.38221274)(82.36183334,4.74125537)(82.41714673,5.18665784)
\lineto(83.82726065,5.18665784)
\curveto(83.79038174,4.77230755)(83.69430378,4.39773386)(83.54000068,4.06393043)
\curveto(83.38569758,3.7291532)(83.18092075,3.44482458)(82.92762794,3.20902679)
\curveto(82.67433513,2.97322899)(82.37736601,2.79370271)(82.0367305,2.66949401)
\curveto(81.69705896,2.54528531)(81.3224563,2.48318096)(80.91485045,2.48318096)
\curveto(80.40729093,2.48318096)(79.95115915,2.57148837)(79.5455011,2.74809327)
\curveto(79.140807,2.92373431)(78.79823364,3.1673026)(78.519699,3.47587676)
\curveto(78.2402004,3.78542471)(78.02669825,4.14931646)(77.87820872,4.56658813)
\curveto(77.72972913,4.98482367)(77.65499741,5.43411138)(77.65499741,5.91736271)
\curveto(77.65499741,6.41225986)(77.72972913,6.87124578)(77.87820872,7.29530422)
\curveto(78.02668831,7.71936266)(78.2402004,8.08907731)(78.519699,8.40444817)
\curveto(78.7982237,8.71981902)(79.140807,8.96727256)(79.5455011,9.14678891)
\curveto(79.95115915,9.32630526)(80.40729093,9.41558648)(80.91485045,9.41558648)
\curveto(81.27975389,9.41558648)(81.62427504,9.36319028)(81.94938781,9.25741415)
\curveto(82.27352669,9.15261182)(82.5646723,8.9992886)(82.82087687,8.79745443)
\curveto(83.07805533,8.59658412)(83.28961962,8.34720286)(83.45654364,8.05026457)
\curveto(83.62346766,7.75332629)(83.72828087,7.41272616)(83.77194725,7.02942805)
\lineto(82.36183334,7.02942805)
\curveto(82.33757534,7.19633473)(82.28128805,7.34965795)(82.19491926,7.48841397)
\closepath
}
}
{
\newrgbcolor{curcolor}{1 1 1}
\pscustom[linestyle=none,fillstyle=solid,fillcolor=curcolor]
{
\newpath
\moveto(98.35537427,9.25257498)
\lineto(101.12221663,4.8101763)
\lineto(101.13773936,4.8101763)
\lineto(101.13773936,9.25257498)
\lineto(102.50417696,9.25257498)
\lineto(102.50417696,2.63070114)
\lineto(101.04748491,2.63070114)
\lineto(98.29131568,7.06436547)
\lineto(98.27288119,7.06436547)
\lineto(98.27288119,2.63070114)
\lineto(96.90644359,2.63070114)
\lineto(96.90644359,9.25257498)
\lineto(98.35537427,9.25257498)
\closepath
}
}
{
\newrgbcolor{curcolor}{1 1 1}
\pscustom[linestyle=none,fillstyle=solid,fillcolor=curcolor]
{
\newpath
\moveto(106.64036867,9.25257498)
\curveto(107.06834686,9.25257498)(107.46527959,9.18464773)(107.83405874,9.04879322)
\curveto(108.20283789,8.91293871)(108.52116319,8.70819309)(108.79095263,8.43648407)
\curveto(109.05977811,8.16477506)(109.27036851,7.82417493)(109.42176979,7.41661141)
\curveto(109.57413504,7.00808402)(109.64983071,6.52871793)(109.64983071,5.9785032)
\curveto(109.64983071,5.49622566)(109.58771991,5.05178706)(109.46447221,4.64325968)
\curveto(109.34025061,4.23569616)(109.15294431,3.88248636)(108.90256326,3.58651193)
\curveto(108.6512083,3.28957365)(108.33871645,3.05571351)(107.96410385,2.88589537)
\curveto(107.58949126,2.71607724)(107.14890209,2.63068127)(106.64036867,2.63068127)
\lineto(103.78036011,2.63068127)
\lineto(103.78036011,9.25255511)
\lineto(106.64036867,9.25255511)
\lineto(106.64036867,9.25257498)
\closepath
\moveto(106.53846721,3.85725709)
\curveto(106.7490576,3.85725709)(106.95286053,3.89122072)(107.15083994,3.95914797)
\curveto(107.34881936,4.02707522)(107.5254426,4.13963812)(107.67975564,4.29781044)
\curveto(107.83405874,4.45500897)(107.95828034,4.65976453)(108.05144654,4.91303104)
\curveto(108.14363885,5.16629755)(108.19022691,5.47585544)(108.19022691,5.83974718)
\curveto(108.19022691,6.17356055)(108.15819762,6.47437414)(108.09317506,6.74317171)
\curveto(108.02815251,7.01196927)(107.92140143,7.24194416)(107.7729119,7.43311626)
\curveto(107.62442237,7.62428835)(107.42839075,7.77177873)(107.1838332,7.87366961)
\curveto(106.93927565,7.97556049)(106.63745691,8.02601903)(106.27934095,8.02601903)
\lineto(105.23995398,8.02601903)
\lineto(105.23995398,3.85726703)
\lineto(106.53846721,3.85726703)
\lineto(106.53846721,3.85725709)
\closepath
}
}
{
\newrgbcolor{curcolor}{1 1 1}
\pscustom[linestyle=none,fillstyle=solid,fillcolor=curcolor]
{
\newpath
\moveto(87.80140554,27.01876818)
\curveto(87.80623779,21.51818141)(83.34980207,17.05685254)(77.84815396,17.05246523)
\curveto(72.3474404,17.04909212)(67.88516952,21.50457504)(67.881306,27.00422736)
\lineto(67.881306,27.01876818)
\curveto(67.87740829,32.51984496)(72.3338782,36.98117383)(77.83455758,36.98554975)
\curveto(83.33523695,36.98895704)(87.79750783,32.53343993)(87.80140554,27.03382179)
\lineto(87.80140554,27.01876818)
\closepath
}
}
{
\newrgbcolor{curcolor}{0 0 0}
\pscustom[linestyle=none,fillstyle=solid,fillcolor=curcolor]
{
\newpath
\moveto(77.82581623,38.61383593)
\curveto(81.07788316,38.61383593)(83.82531097,37.49305375)(86.07007129,35.25197942)
\curveto(88.31478602,33.00989087)(89.43763345,30.26517564)(89.43763345,27.01877957)
\curveto(89.43763345,23.77286212)(88.33419478,21.0572627)(86.12828617,18.87293856)
\curveto(83.78649345,16.5731198)(81.01867676,15.42420184)(77.82581623,15.42420184)
\curveto(74.67270775,15.42420184)(71.95344145,16.56439075)(69.669875,18.84480277)
\curveto(67.38634275,21.12420059)(66.24505529,23.84950907)(66.24505529,27.01877957)
\curveto(66.24505529,30.18854009)(67.38633136,32.93324392)(69.669875,35.25197942)
\curveto(71.89519238,37.49305375)(74.61449287,38.61383593)(77.82581623,38.61383593)
\closepath
\moveto(68.853727,30.13417158)
\curveto(68.50822145,29.15556118)(68.33452275,28.11726052)(68.33452275,27.01876818)
\curveto(68.33452275,24.44873643)(69.2739386,22.22702326)(71.15085564,20.35177117)
\curveto(73.02773849,18.47796632)(75.26278873,17.54059668)(77.85590379,17.54059668)
\curveto(80.44803873,17.54059668)(82.70246354,18.48718538)(84.6181981,20.3813656)
\curveto(85.25969007,21.00094554)(85.78862719,21.67683125)(86.20299223,22.40946716)
\lineto(81.83097265,24.35556583)
\curveto(81.53499762,22.88496087)(80.22387497,21.89177547)(78.64004957,21.77531222)
\lineto(78.64004957,19.98740756)
\lineto(77.30852664,19.98740756)
\lineto(77.30852664,21.77531222)
\curveto(76.00711407,21.78988722)(74.75036558,22.32214251)(73.78862338,23.16345846)
\lineto(75.38601098,24.77379654)
\curveto(76.15465711,24.04988968)(76.92423777,23.72531958)(77.9742938,23.72531958)
\curveto(78.65459189,23.72531958)(79.4086729,23.99119083)(79.4086729,24.87764484)
\curveto(79.4086729,25.19153724)(79.28735391,25.40990014)(79.0961725,25.57436175)
\lineto(77.98985044,26.06537629)
\lineto(76.61369762,26.67914447)
\curveto(75.93241941,26.98288337)(75.35496609,27.23856693)(74.77656683,27.49667776)
\lineto(68.853727,30.13417158)
\closepath
\moveto(77.85590379,36.52705831)
\curveto(75.22300248,36.52705831)(72.99863105,35.59984216)(71.17899435,33.74398542)
\curveto(70.6840311,33.24424183)(70.2531407,32.72266423)(69.88826058,32.17778258)
\lineto(74.32045529,30.20402674)
\curveto(74.72125814,31.43401315)(75.88970418,32.18020985)(77.30853804,32.26270276)
\lineto(77.30853804,34.05064159)
\lineto(78.64006097,34.05064159)
\lineto(78.64006097,32.26270276)
\curveto(79.55715048,32.21853333)(80.56257662,31.96723708)(81.55342625,31.1996508)
\lineto(80.02977598,29.63344797)
\curveto(79.46786791,30.03230612)(78.75845098,30.31274097)(78.04806532,30.31274097)
\curveto(77.47158073,30.31274097)(76.65733599,30.13610883)(76.65733599,29.41171196)
\curveto(76.65733599,29.30157328)(76.69520757,29.20455096)(76.7621638,29.11722631)
\lineto(78.2450591,28.45834286)
\lineto(79.24853639,28.01097355)
\curveto(79.89002836,27.72423692)(80.50337022,27.45253112)(81.10990819,27.18228395)
\lineto(87.05219097,24.53702971)
\curveto(87.24821602,25.31478088)(87.34719727,26.14252465)(87.34719727,27.01876818)
\curveto(87.34719727,29.66646108)(86.41846023,31.90850405)(84.56095195,33.74398542)
\curveto(82.72191789,35.59984216)(80.48783637,36.52705831)(77.85590379,36.52705831)
\closepath
}
}
{
\newrgbcolor{curcolor}{1 1 1}
\pscustom[linestyle=none,fillstyle=solid,fillcolor=curcolor]
{
\newpath
\moveto(114.66821399,27.01876403)
\curveto(114.67306481,21.1741642)(109.93711693,16.43192018)(104.09191009,16.42706987)
\curveto(98.2457157,16.42369763)(93.50298538,21.15816625)(93.49907864,27.00420689)
\lineto(93.49907864,27.01876403)
\curveto(93.49520917,32.86385448)(98.2301757,37.6060985)(104.07637009,37.61049545)
\curveto(109.92157693,37.61534576)(114.66431347,32.8798897)(114.66821399,27.03479924)
\lineto(114.66821399,27.01876403)
\closepath
}
}
{
\newrgbcolor{curcolor}{0 0 0}
\pscustom[linestyle=none,fillstyle=solid,fillcolor=curcolor]
{
\newpath
\moveto(103.91816273,38.6138453)
\curveto(100.70585512,38.6138453)(97.98659773,37.49405073)(95.7612601,35.25195879)
\curveto(93.47773133,32.93326175)(92.33643588,30.18949488)(92.33643588,27.01876403)
\curveto(92.33643588,23.84951125)(93.47772512,21.12515184)(95.7612601,18.84477908)
\curveto(98.04384479,16.56440012)(100.76408974,15.42417959)(103.91816273,15.42417959)
\curveto(107.11106087,15.42417959)(109.87789476,16.57456652)(112.21965194,18.87290592)
\curveto(114.42459875,21.05722885)(115.5290069,23.77286888)(115.5290069,27.01877024)
\curveto(115.5290069,30.26664029)(114.40617684,33.00991032)(112.16142353,35.251965)
\curveto(109.9167137,37.49405073)(107.16830173,38.6138453)(103.91816273,38.6138453)
\closepath
\moveto(103.94632975,36.52803771)
\curveto(106.57824764,36.52803771)(108.81329311,35.59985215)(110.65234191,33.74401513)
\curveto(112.50984991,31.90852954)(113.43764119,29.66742505)(113.43764119,27.01876403)
\curveto(113.43764119,24.35217362)(112.52926559,22.13972386)(110.71057032,20.38137746)
\curveto(108.79388364,18.48717889)(106.5394287,17.54156078)(103.94632975,17.54156078)
\curveto(101.35319354,17.54156078)(99.11818534,18.47796889)(97.24126788,20.35325063)
\curveto(95.36435042,22.22705429)(94.42590722,24.44970152)(94.42590722,27.01876403)
\curveto(94.42590722,29.59029207)(95.37405205,31.83139655)(97.27132927,33.74401513)
\curveto(99.09002453,35.59985215)(101.3143746,36.52803771)(103.94632975,36.52803771)
\closepath
}
}
{
\newrgbcolor{curcolor}{0 0 0}
\pscustom[linestyle=none,fillstyle=solid,fillcolor=curcolor]
{
\newpath
\moveto(108.31346044,29.76738742)
\lineto(99.8848625,29.76738742)
\lineto(99.8848625,27.77081064)
\lineto(108.31346044,27.77081064)
\lineto(108.31346044,29.76738742)
\closepath
\moveto(108.31346044,26.04062213)
\lineto(99.8848625,26.04062213)
\lineto(99.8848625,24.04457324)
\lineto(108.31346044,24.04457324)
\lineto(108.31346044,26.04062213)
\closepath
}
}
{
\newrgbcolor{curcolor}{1 1 1}
\pscustom[linestyle=none,fillstyle=solid,fillcolor=curcolor]
{
\newpath
\moveto(62.50291032,26.78298342)
\curveto(62.50291032,20.85269161)(57.69495965,16.04524024)(51.76405194,16.04524024)
\curveto(45.83314423,16.04524024)(41.02519356,20.85269161)(41.02519356,26.78298342)
\curveto(41.02519356,32.71327522)(45.83314423,37.5207266)(51.76405194,37.5207266)
\curveto(57.69495965,37.5207266)(62.50291032,32.71327522)(62.50291032,26.78298342)
\closepath
}
}
{
\newrgbcolor{curcolor}{0 0 0}
\pscustom[linestyle=none,fillstyle=solid,fillcolor=curcolor]
{
\newpath
\moveto(54.87202058,29.89013865)
\curveto(54.87202058,30.30400204)(54.53623469,30.6387892)(54.12280533,30.6387892)
\lineto(49.38006439,30.6387892)
\curveto(48.96663503,30.6387892)(48.63084914,30.30401197)(48.63084914,29.89013865)
\lineto(48.63084914,25.14740333)
\lineto(49.95361042,25.14740333)
\lineto(49.95361042,19.5308449)
\lineto(53.54827546,19.5308449)
\lineto(53.54827546,25.14740333)
\lineto(54.87201064,25.14740333)
\lineto(54.87201064,29.89013865)
\lineto(54.87202058,29.89013865)
\closepath
}
}
{
\newrgbcolor{curcolor}{0 0 0}
\pscustom[linestyle=none,fillstyle=solid,fillcolor=curcolor]
{
\newpath
\moveto(53.37359095,32.88570429)
\curveto(53.37359095,31.98990571)(52.6473273,31.26371748)(51.75143568,31.26371748)
\curveto(50.85554407,31.26371748)(50.12928042,31.98990571)(50.12928042,32.88570429)
\curveto(50.12928042,33.78150287)(50.85554407,34.5076911)(51.75143568,34.5076911)
\curveto(52.6473273,34.5076911)(53.37359095,33.78150287)(53.37359095,32.88570429)
\closepath
}
}
{
\newrgbcolor{curcolor}{0 0 0}
\pscustom[linestyle=none,fillstyle=solid,fillcolor=curcolor]
{
\newpath
\moveto(51.73638914,38.61386217)
\curveto(48.52506202,38.61386217)(45.80578164,37.49355931)(43.58046596,35.25149291)
\curveto(41.29692513,32.9332518)(40.15563669,30.18853763)(40.15563669,27.01975508)
\curveto(40.15563669,23.85097254)(41.29692513,21.12517784)(43.58046596,18.8447856)
\curveto(45.86400678,16.56488026)(48.58329711,15.42468414)(51.73638914,15.42468414)
\curveto(54.92927183,15.42468414)(57.69708808,16.57410152)(60.03885406,18.87438702)
\curveto(62.24476136,21.05726052)(63.34722309,23.77287011)(63.34722309,27.01976502)
\curveto(63.34722309,30.26665993)(62.22534304,33.00992334)(59.98061897,35.25150285)
\curveto(57.73590485,37.49354938)(54.98848081,38.61386217)(51.73638914,38.61386217)
\closepath
\moveto(51.76648058,36.52754354)
\curveto(54.39842821,36.52754354)(56.63246922,35.59986354)(58.47056133,33.7444936)
\curveto(60.32903572,31.90901692)(61.25778596,29.66695052)(61.25778596,27.01975508)
\curveto(61.25778596,24.35315328)(60.3484441,22.1397145)(58.52879641,20.38138633)
\curveto(56.61307077,18.48769056)(54.35863755,17.54108115)(51.76648058,17.54108115)
\curveto(49.17238575,17.54108115)(46.93736091,18.47798241)(45.06045204,20.35227182)
\curveto(43.18354316,22.22704812)(42.24508376,24.44922126)(42.24508376,27.01975508)
\curveto(42.24508376,29.58981195)(43.19324238,31.83139145)(45.08955964,33.7444936)
\curveto(46.90921726,35.5998536)(49.13453294,36.52754354)(51.76648058,36.52754354)
\closepath
}
}
\end{pspicture}
 \\ {\tt cc-by-cs-nd}

        \vspace{3pt}%LaTeX with PSTricks extensions
%%Creator: inkscape 0.91
%%Please note this file requires PSTricks extensions
\psset{xunit=.5pt,yunit=.5pt,runit=.5pt}
\begin{pspicture}(120,42)
{
\newrgbcolor{curcolor}{0.66666669 0.69803923 0.67058825}
\pscustom[linestyle=none,fillstyle=solid,fillcolor=curcolor]
{
\newpath
\moveto(3.4078519,41.56089114)
\lineto(116.76243,41.35952925)
\curveto(118.34626,41.35952925)(119.76124,41.59436864)(119.76124,38.1994076)
\lineto(119.62246,0.869463)
\lineto(0.54733338,0.869463)
\lineto(0.54733338,38.3376899)
\curveto(0.54733338,40.0121316)(0.70940909,41.56089114)(3.4078519,41.56089114)
\closepath
}
}
{
\newrgbcolor{curcolor}{1 1 1}
\pscustom[linestyle=none,fillstyle=solid,fillcolor=curcolor]
{
\newpath
\moveto(34.5222076,22.46190692)
\curveto(34.5270482,14.93253481)(28.42559457,8.82481283)(20.89557229,8.81997266)
\curveto(13.36555001,8.81560089)(7.25680944,14.91604525)(7.25243728,22.44539134)
\lineto(7.25243728,22.46190692)
\curveto(7.24806512,29.9917301)(13.34904164,36.09945207)(20.87907259,36.10381517)
\curveto(28.41004911,36.10818694)(34.51783544,30.00774258)(34.5222076,22.47839649)
\lineto(34.5222076,22.46190692)
\closepath
}
}
{
\newrgbcolor{curcolor}{0 0 0}
\pscustom[linestyle=none,fillstyle=solid,fillcolor=curcolor]
{
\newpath
\moveto(31.97127768,33.5666951)
\curveto(34.99484332,30.54342351)(36.5068647,26.84131279)(36.5068647,22.4619156)
\curveto(36.5068647,18.08198061)(35.0209982,14.41921584)(32.04936931,11.47355189)
\curveto(28.89576661,8.37165779)(25.16863018,6.82093626)(20.86789062,6.82093626)
\curveto(16.61908781,6.82093626)(12.95647544,8.35855114)(9.8809904,11.43522081)
\curveto(6.80502823,14.51043322)(5.26730306,18.18630464)(5.26730306,22.4619156)
\curveto(5.26730306,26.73702346)(6.80502823,30.43862241)(9.8809904,33.5666951)
\curveto(12.8783491,36.59144997)(16.54097015,38.10286024)(20.86789062,38.10286024)
\curveto(25.24720761,38.10286891)(28.94771204,36.59144997)(31.97127768,33.5666951)
\closepath
\moveto(11.91661266,31.53321365)
\curveto(9.36030429,28.95145639)(8.08267927,25.92672755)(8.08267927,22.4580209)
\curveto(8.08267927,18.98979134)(9.34769963,15.99124108)(11.87727189,13.46189306)
\curveto(14.40736465,10.9325537)(17.41926252,9.66761079)(20.91398047,9.66761079)
\curveto(24.40872445,9.66761079)(27.44637816,10.94467149)(30.02836429,13.50070121)
\curveto(32.4797842,15.87383644)(33.70599295,18.85877695)(33.70599295,22.4580209)
\curveto(33.70599295,26.03007139)(32.45987525,29.06160945)(29.96911458,31.55214065)
\curveto(27.47888309,34.04220344)(24.46066987,35.28770758)(20.91398047,35.28770758)
\curveto(17.36735179,35.28770758)(14.36755544,34.03590601)(11.91661266,31.53321365)
\closepath
\moveto(18.64354545,23.98349197)
\curveto(18.25292243,24.83550188)(17.66819805,25.26102975)(16.88841808,25.26102975)
\curveto(15.50985169,25.26102975)(14.82079404,24.33283477)(14.82079404,22.47790206)
\curveto(14.82079404,20.6220152)(15.50985169,19.69429729)(16.88841808,19.69429729)
\curveto(17.79872092,19.69429729)(18.44896697,20.14602978)(18.83908684,21.05140307)
\lineto(20.74998976,20.03395106)
\curveto(19.83918378,18.41580535)(18.47274493,17.6064983)(16.65066452,17.6064983)
\curveto(15.24538804,17.6064983)(14.11961752,18.03734343)(13.27430719,18.89856528)
\curveto(12.42757418,19.76029024)(12.00541782,20.94805922)(12.00541782,22.46190692)
\curveto(12.00541782,23.94952399)(12.44115043,25.13099554)(13.31314482,26.00531536)
\curveto(14.18511317,26.88013829)(15.27111786,27.31728953)(16.5730153,27.31728953)
\curveto(18.49895186,27.31728953)(19.87802141,26.55893444)(20.71165529,25.042684)
\lineto(18.64354545,23.98349197)
\closepath
\moveto(27.63366103,23.98349197)
\curveto(27.24253486,24.83550188)(26.66948693,25.26102975)(25.91397939,25.26102975)
\curveto(24.50769662,25.26102975)(23.80412583,24.33283477)(23.80412583,22.47790206)
\curveto(23.80412583,20.6220152)(24.50769662,19.69429729)(25.91397939,19.69429729)
\curveto(26.82573961,19.69429729)(27.46433524,20.14602978)(27.82871663,21.05140307)
\lineto(29.78231751,20.03395106)
\curveto(28.87299494,18.41580535)(27.50849059,17.6064983)(25.6897934,17.6064983)
\curveto(24.28641673,17.6064983)(23.16308386,18.03734343)(22.31831138,18.89856528)
\curveto(21.47496159,19.76029024)(21.05228474,20.94805922)(21.05228474,22.46190692)
\curveto(21.05228474,23.94952399)(21.48126826,25.13099554)(22.33869745,26.00531536)
\curveto(23.19559747,26.88013829)(24.28642541,27.31728953)(25.61212684,27.31728953)
\curveto(27.53465415,27.31728953)(28.91181521,26.55893444)(29.74256035,25.042684)
\lineto(27.63366103,23.98349197)
\closepath
}
}
{
\newrgbcolor{curcolor}{0 0 0}
\pscustom[linestyle=none,fillstyle=solid,fillcolor=curcolor]
{
\newpath
\moveto(117.7533,41.999994)
\lineto(2.2471557,41.999994)
\curveto(1.0083284,41.999994)(-0.00001127,40.9927175)(-0.00001127,39.7540001)
\lineto(-0.00001127,0.507024)
\curveto(-0.00001127,0.227061)(0.22756453,-0.000008)(0.50755234,-0.000008)
\lineto(119.49242,-0.000008)
\curveto(119.77241,-0.000008)(119.99998,0.227071)(119.99998,0.507024)
\lineto(119.99998,39.7540001)
\curveto(119.99998,40.9927175)(118.99213,41.999994)(117.7533,41.999994)
\closepath
\moveto(2.2471458,40.9859306)
\lineto(117.75329,40.9859306)
\curveto(118.43263,40.9859306)(118.98485,40.4332884)(118.98485,39.7540001)
\lineto(118.98485,12.545352)
\lineto(36.427769,12.545352)
\curveto(33.401786,7.07422)(27.572059,3.360005)(20.88151,3.360005)
\curveto(14.188048,3.360005)(8.3597722,7.071308)(5.33525,12.545352)
\lineto(1.0151159,12.545352)
\lineto(1.0151159,39.7540001)
\curveto(1.0151159,40.4332785)(1.5678071,40.9859306)(2.2471458,40.9859306)
\closepath
}
}
{
\newrgbcolor{curcolor}{1 1 1}
\pscustom[linestyle=none,fillstyle=solid,fillcolor=curcolor]
{
\newpath
\moveto(59.99658281,9.25278152)
\curveto(60.31247592,9.25278152)(60.60022515,9.22464014)(60.86031746,9.1693312)
\curveto(61.12040977,9.11402227)(61.34362288,9.02280128)(61.52995678,8.89567817)
\curveto(61.71531677,8.76952888)(61.85992181,8.60067068)(61.9618241,8.39107108)
\curveto(62.06372638,8.18049766)(62.11516447,7.92139229)(62.11516447,7.61183714)
\curveto(62.11516447,7.27801601)(62.03946819,7.00048758)(61.88710172,6.77825814)
\curveto(61.73570915,6.55506483)(61.51054824,6.37359667)(61.21358666,6.23095208)
\curveto(61.62313366,6.11353735)(61.92884051,5.90780321)(62.13069727,5.61377948)
\curveto(62.33255402,5.31975575)(62.43349234,4.96555398)(62.43349234,4.55118413)
\curveto(62.43349234,4.217363)(62.36846926,3.92818849)(62.23842311,3.68364072)
\curveto(62.10837695,3.44006676)(61.93271626,3.2401656)(61.71338884,3.08587494)
\curveto(61.49308751,2.93061045)(61.24173053,2.81610724)(60.96029181,2.74235537)
\curveto(60.67787918,2.66762968)(60.38866909,2.63076368)(60.09073361,2.63076368)
\lineto(56.87453099,2.63076368)
\lineto(56.87453099,9.25279145)
\lineto(59.99658281,9.25279145)
\lineto(59.99658281,9.25278152)
\closepath
\moveto(59.81073587,6.57446171)
\curveto(60.07034123,6.57446171)(60.28433204,6.6365675)(60.45174436,6.75980527)
\curveto(60.61866972,6.88304304)(60.70212743,7.08295414)(60.70212743,7.36048257)
\curveto(60.70212743,7.51477324)(60.67398356,7.64189634)(60.61866972,7.73991419)
\curveto(60.56238197,7.83889592)(60.48862356,7.91555931)(60.39545661,7.97184207)
\curveto(60.30277662,8.027151)(60.19602468,8.06597458)(60.07519088,8.0873191)
\curveto(59.95435707,8.10866362)(59.8291705,8.11934581)(59.69912435,8.11934581)
\lineto(58.33412667,8.11934581)
\lineto(58.33412667,6.57446171)
\lineto(59.81073587,6.57446171)
\closepath
\moveto(59.89565444,3.76417951)
\curveto(60.03831167,3.76417951)(60.17418138,3.77776329)(60.30471449,3.80590467)
\curveto(60.43427369,3.83404604)(60.54976096,3.88063036)(60.64875146,3.94467385)
\curveto(60.74774197,4.00969117)(60.82635003,4.09800064)(60.88554955,4.20959233)
\curveto(60.94474908,4.32022013)(60.97386685,4.46286472)(60.97386685,4.63559832)
\curveto(60.97386685,4.97524249)(60.87778813,5.21784262)(60.68563067,5.36339873)
\curveto(60.49347321,5.50799096)(60.23969141,5.58076901)(59.92379831,5.58076901)
\lineto(58.3341366,5.58076901)
\lineto(58.3341366,3.76417951)
\lineto(59.89565444,3.76417951)
\lineto(59.89565444,3.76417951)
\closepath
}
}
{
\newrgbcolor{curcolor}{1 1 1}
\pscustom[linestyle=none,fillstyle=solid,fillcolor=curcolor]
{
\newpath
\moveto(62.69066294,9.25278152)
\lineto(64.32399137,9.25278152)
\lineto(65.874836,6.63754132)
\lineto(67.41597139,9.25278152)
\lineto(69.03960052,9.25278152)
\lineto(66.58135452,5.17223213)
\lineto(66.58135452,2.63074381)
\lineto(65.12173897,2.63074381)
\lineto(65.12173897,5.20910806)
\lineto(62.69066294,9.25278152)
\closepath
}
}
{
\newrgbcolor{curcolor}{1 1 1}
\pscustom[linestyle=none,fillstyle=solid,fillcolor=curcolor]
{
\newpath
\moveto(86.4336763,9.25278152)
\lineto(89.20054093,4.81027956)
\lineto(89.21606378,4.81027956)
\lineto(89.21606378,9.25278152)
\lineto(90.58251238,9.25278152)
\lineto(90.58251238,2.63075374)
\lineto(89.12580861,2.63075374)
\lineto(86.36961719,7.06452114)
\lineto(86.35118255,7.06452114)
\lineto(86.35118255,2.63075374)
\lineto(84.98473395,2.63075374)
\lineto(84.98473395,9.25278152)
\lineto(86.4336763,9.25278152)
\closepath
}
}
{
\newrgbcolor{curcolor}{1 1 1}
\pscustom[linestyle=none,fillstyle=solid,fillcolor=curcolor]
{
\newpath
\moveto(96.11139199,7.48858943)
\curveto(96.0250225,7.6283225)(95.91632275,7.75059639)(95.7862766,7.85540116)
\curveto(95.65623044,7.96020593)(95.50968753,8.04269236)(95.34567394,8.10091281)
\curveto(95.18166036,8.16010709)(95.00988535,8.18922228)(94.83131288,8.18922228)
\curveto(94.50328571,8.18922228)(94.22475876,8.12614267)(93.9957221,7.99901956)
\curveto(93.76668544,7.87287027)(93.58132545,7.70304819)(93.43963218,7.48955332)
\curveto(93.29697495,7.27606838)(93.19312485,7.03346824)(93.12810178,6.76175291)
\curveto(93.0630787,6.49003758)(93.03104914,6.2086238)(93.03104914,5.91847547)
\curveto(93.03104914,5.63997322)(93.0630787,5.36923171)(93.12810178,5.10721482)
\curveto(93.19312485,4.84423404)(93.29696501,4.60745695)(93.43963218,4.39784741)
\curveto(93.58132545,4.18727399)(93.76668544,4.01938961)(93.9957221,3.8922665)
\curveto(94.22475876,3.76514339)(94.50328571,3.70206378)(94.83131288,3.70206378)
\curveto(95.27580121,3.70206378)(95.62420089,3.83792144)(95.87458396,4.11060066)
\curveto(96.12496703,4.38231599)(96.27830741,4.74136697)(96.33362125,5.18677979)
\lineto(97.74374652,5.18677979)
\curveto(97.70686731,4.77241987)(97.61078858,4.39783747)(97.45648424,4.06402628)
\curveto(97.30217989,3.72924127)(97.09740142,3.44490605)(96.84410657,3.20910277)
\curveto(96.59081172,2.97329949)(96.2938402,2.79376903)(95.95320196,2.66955744)
\curveto(95.61352768,2.54534585)(95.23892201,2.48324006)(94.83131288,2.48324006)
\curveto(94.32374927,2.48324006)(93.86761382,2.57154953)(93.4619525,2.74815853)
\curveto(93.05725514,2.92380366)(92.71467903,3.16737761)(92.43614214,3.47595894)
\curveto(92.15664129,3.78551409)(91.94313743,4.1494143)(91.7946467,4.56669567)
\curveto(91.64616591,4.98494093)(91.57143359,5.43423908)(91.57143359,5.91750165)
\curveto(91.57143359,6.4124103)(91.64616591,6.8714069)(91.7946467,7.29547519)
\curveto(91.94312749,7.71954349)(92.15664129,8.08926673)(92.43614214,8.40464492)
\curveto(92.71466909,8.72002311)(93.05725514,8.96748241)(93.4619525,9.14700293)
\curveto(93.86761382,9.32652345)(94.32374927,9.41580674)(94.83131288,9.41580674)
\curveto(95.19621925,9.41580674)(95.54074318,9.36340932)(95.86585856,9.25763074)
\curveto(96.19000005,9.15282597)(96.48114801,8.99949918)(96.73735464,8.79766032)
\curveto(96.99453517,8.59678534)(97.20610116,8.34739829)(97.37302653,8.0504531)
\curveto(97.53995189,7.75350791)(97.64476595,7.41289986)(97.68843268,7.02959284)
\lineto(96.27830741,7.02959284)
\curveto(96.25405916,7.19649346)(96.19777141,7.34982025)(96.11139199,7.48858943)
\closepath
}
}
{
\newrgbcolor{curcolor}{1 1 1}
\pscustom[linestyle=none,fillstyle=solid,fillcolor=curcolor]
{
\newpath
\moveto(74.09876581,26.78310848)
\curveto(74.09876581,20.85254169)(69.29066527,16.04486739)(63.35957268,16.04486739)
\curveto(57.42848009,16.04486739)(52.62037954,20.85254169)(52.62037954,26.78310848)
\curveto(52.62037954,32.71367527)(57.42848009,37.52134957)(63.35957268,37.52134957)
\curveto(69.29066527,37.52134957)(74.09876581,32.71367527)(74.09876581,26.78310848)
\closepath
}
}
{
\newrgbcolor{curcolor}{0 0 0}
\pscustom[linestyle=none,fillstyle=solid,fillcolor=curcolor]
{
\newpath
\moveto(66.46780513,29.88985113)
\curveto(66.46780513,30.30372414)(66.13201653,30.639006)(65.71858384,30.639006)
\lineto(60.97580473,30.639006)
\curveto(60.56237204,30.639006)(60.22658344,30.30373408)(60.22658344,29.88985113)
\lineto(60.22658344,25.14700556)
\lineto(61.54935538,25.14700556)
\lineto(61.54935538,19.53079354)
\lineto(65.14404935,19.53079354)
\lineto(65.14404935,25.14700556)
\lineto(66.46779519,25.14700556)
\lineto(66.46779519,29.88985113)
\lineto(66.46780513,29.88985113)
\closepath
}
}
{
\newrgbcolor{curcolor}{0 0 0}
\pscustom[linestyle=none,fillstyle=solid,fillcolor=curcolor]
{
\newpath
\moveto(64.96936726,32.88548792)
\curveto(64.96936726,31.98966851)(64.24309777,31.2634634)(63.34719894,31.2634634)
\curveto(62.45130012,31.2634634)(61.72503062,31.98966851)(61.72503062,32.88548792)
\curveto(61.72503062,33.78130732)(62.45130012,34.50751243)(63.34719894,34.50751243)
\curveto(64.24309777,34.50751243)(64.96936726,33.78130732)(64.96936726,32.88548792)
\closepath
}
}
{
\newrgbcolor{curcolor}{0 0 0}
\pscustom[linestyle=none,fillstyle=solid,fillcolor=curcolor]
{
\newpath
\moveto(63.33215838,38.61426436)
\curveto(60.12080541,38.61426436)(57.40101618,37.49393546)(55.17568259,35.25181694)
\curveto(52.89212338,32.93303503)(51.75082576,30.18825705)(51.75082576,27.01940084)
\curveto(51.75082576,23.85054463)(52.89212338,21.12517347)(55.17568259,18.84472822)
\curveto(57.4592418,16.56476988)(60.17855402,15.42454725)(63.33215838,15.42454725)
\curveto(66.52506677,15.42454725)(69.29290531,16.57350444)(71.63372617,18.87384342)
\curveto(73.84061519,21.05725458)(74.94309574,23.77244038)(74.94309574,27.01941078)
\curveto(74.94309574,30.26638117)(73.82120665,33.01019526)(71.57646452,35.25182688)
\curveto(69.33172239,37.49393546)(66.58329239,38.61426436)(63.33215838,38.61426436)
\closepath
\moveto(63.36126622,36.52789723)
\curveto(65.99323504,36.52789723)(68.22826794,35.59970876)(70.06637484,33.74478259)
\curveto(71.92486419,31.90877634)(72.85362191,29.66714472)(72.85362191,27.01940084)
\curveto(72.85362191,24.35273705)(71.94427273,22.13973372)(70.1246104,20.38135474)
\curveto(68.20886934,18.48712803)(65.95441796,17.54049662)(63.36127616,17.54049662)
\curveto(60.76813435,17.54049662)(58.53309152,18.47790656)(56.65665449,20.35223954)
\curveto(54.7787566,22.22705943)(53.84077659,24.44928423)(53.84077659,27.0193909)
\curveto(53.84077659,29.58998449)(54.78894285,31.8316161)(56.68527538,33.74477266)
\curveto(58.50446069,35.59970876)(60.7302713,36.52789723)(63.36126622,36.52789723)
\closepath
}
}
{
\newrgbcolor{curcolor}{1 1 1}
\pscustom[linestyle=none,fillstyle=solid,fillcolor=curcolor]
{
\newpath
\moveto(101.717749,27.01916325)
\curveto(101.722581,21.51857625)(97.266146,17.05724625)(91.764498,17.05285925)
\curveto(86.263784,17.04948625)(81.801513,21.50496925)(81.79765,27.00462225)
\lineto(81.79765,27.01916325)
\curveto(81.793752,32.52023975)(86.250222,36.98156875)(91.750901,36.98594465)
\curveto(97.251581,36.98935195)(101.713851,32.53383475)(101.717749,27.03421625)
\lineto(101.717749,27.01916325)
\closepath
}
}
{
\newrgbcolor{curcolor}{0 0 0}
\pscustom[linestyle=none,fillstyle=solid,fillcolor=curcolor]
{
\newpath
\moveto(99.986719,35.25481965)
\curveto(97.741599,37.49711165)(94.993939,38.61755165)(91.743039,38.61755165)
\curveto(88.530359,38.61755165)(85.811719,37.49711165)(83.586419,35.25481965)
\curveto(81.302369,32.93537765)(80.161399,30.19125725)(80.161399,27.02104925)
\curveto(80.161399,23.85154725)(81.302369,21.12582925)(83.586419,18.84602625)
\curveto(85.869759,16.56480625)(88.588399,15.42454725)(91.743039,15.42454725)
\curveto(94.935899,15.42454725)(97.703369,16.57471325)(100.044749,18.87433625)
\curveto(102.250949,21.05858825)(103.354399,23.77439425)(103.354399,27.02104925)
\curveto(103.354399,30.26841025)(102.231839,33.01323765)(99.986719,35.25481965)
\closepath
\moveto(98.535029,20.38193925)
\curveto(96.619029,18.48788325)(94.364709,17.54156125)(91.772059,17.54156125)
\curveto(89.178699,17.54156125)(86.943479,18.47868125)(85.066409,20.35362925)
\curveto(83.189339,22.22857725)(82.250809,24.45105025)(82.250809,27.02105225)
\curveto(82.250809,28.10822425)(82.422449,29.13594325)(82.761839,30.10491425)
\lineto(85.834719,28.74488725)
\lineto(85.613179,28.74488725)
\lineto(85.613179,27.36681325)
\lineto(86.700709,27.36681325)
\curveto(86.700709,27.17216925)(86.681249,26.97823225)(86.681249,26.78429925)
\lineto(86.681249,26.45375925)
\lineto(85.613179,26.45375925)
\lineto(85.613179,25.07568525)
\lineto(86.875179,25.07568525)
\curveto(87.049999,24.04655325)(87.438229,23.21206225)(87.962349,22.55169225)
\curveto(89.049519,21.11487125)(90.796709,20.33842025)(92.699269,20.33842025)
\curveto(93.942149,20.33842025)(95.067549,20.70718125)(95.727919,21.07664925)
\lineto(95.261479,23.23117425)
\curveto(94.853789,23.01742225)(93.942149,22.72651625)(93.048909,22.72651625)
\curveto(92.077819,22.72651625)(91.165829,23.01741825)(90.544389,23.71672025)
\curveto(90.253129,24.04655325)(90.039729,24.49317125)(89.904189,25.07568525)
\lineto(94.124759,25.07568525)
\lineto(100.121899,22.42110125)
\curveto(99.706429,21.68498725)(99.179129,21.00479825)(98.535029,20.38193925)
\closepath
\moveto(90.992779,26.45375225)
\lineto(90.967299,26.47321625)
\lineto(91.011179,26.45375225)
\lineto(90.992779,26.45375225)
\closepath
\moveto(94.603229,27.36680625)
\lineto(94.776649,27.36680625)
\lineto(94.776649,28.74488025)
\lineto(91.489649,28.74488025)
\lineto(90.154399,29.33588825)
\curveto(90.269409,29.59423325)(90.405309,29.82532925)(90.563849,30.00616925)
\curveto(91.165829,30.74510625)(92.019789,31.05512025)(92.951949,31.05512025)
\curveto(93.806259,31.05512025)(94.601819,30.80314625)(95.106479,30.58939025)
\lineto(95.650059,32.80266265)
\curveto(94.950759,33.11267765)(93.922329,33.38447065)(92.738189,33.38447065)
\curveto(90.913499,33.38447065)(89.360249,32.64694765)(88.253609,31.40476925)
\curveto(88.008009,31.12094525)(87.794959,30.80243625)(87.601729,30.46552525)
\lineto(83.788129,32.15326265)
\curveto(84.157599,32.70746665)(84.592189,33.23972765)(85.095429,33.74721565)
\curveto(86.915169,35.60305165)(89.139769,36.53097065)(91.772049,36.53097065)
\curveto(94.403629,36.53097065)(96.638849,35.60305165)(98.477699,33.74721565)
\curveto(100.334949,31.91119625)(101.263579,29.66961425)(101.263579,27.02104525)
\curveto(101.263579,26.14833625)(101.164479,25.32516925)(100.969129,24.54942825)
\lineto(94.603229,27.36680625)
\closepath
}
}
\end{pspicture}
 \\ {\tt cc-by-nc}

        \column{0.25\textwidth}
        \centering

        %LaTeX with PSTricks extensions
%%Creator: inkscape 0.91
%%Please note this file requires PSTricks extensions
\psset{xunit=.5pt,yunit=.5pt,runit=.5pt}
\begin{pspicture}(120,42)
{
\newrgbcolor{curcolor}{0.66666669 0.69803923 0.67058825}
\pscustom[linestyle=none,fillstyle=solid,fillcolor=curcolor]
{
\newpath
\moveto(3.40785191,41.56332807)
\lineto(116.76242533,41.3614939)
\curveto(118.34626348,41.3614939)(119.7612384,41.59632784)(119.7612384,38.20096875)
\lineto(119.62245691,0.87236915)
\lineto(0.54733338,0.87236915)
\lineto(0.54733338,38.33972477)
\curveto(0.54733338,40.01363077)(0.70940909,41.56332807)(3.40785191,41.56332807)
\closepath
}
}
{
\newrgbcolor{curcolor}{1 1 1}
\pscustom[linestyle=none,fillstyle=solid,fillcolor=curcolor]
{
\newpath
\moveto(34.52221352,22.46431335)
\curveto(34.52705413,14.93511605)(28.42560028,8.82753588)(20.89557775,8.82269581)
\curveto(13.36555521,8.81785575)(7.25681443,14.91862687)(7.25244227,22.44779815)
\lineto(7.25244227,22.46431335)
\curveto(7.24807011,29.99446479)(13.34904684,36.1010648)(20.87907805,36.10590487)
\curveto(28.41005482,36.10979947)(34.51784136,30.00997381)(34.52221352,22.48080253)
\lineto(34.52221352,22.46431335)
\closepath
}
}
{
\newrgbcolor{curcolor}{0 0 0}
\pscustom[linestyle=none,fillstyle=solid,fillcolor=curcolor]
{
\newpath
\moveto(31.97128351,33.56836664)
\curveto(34.99484926,30.54564231)(36.50687069,26.84361753)(36.50687069,22.46431335)
\curveto(36.50687069,18.08400299)(35.02100414,14.42180032)(32.04937515,11.47570167)
\curveto(28.89577234,8.37438268)(25.16863578,6.82369716)(20.86789607,6.82369716)
\curveto(16.61909312,6.82369716)(12.95648062,8.36079927)(9.88099548,11.43787458)
\curveto(6.80503321,14.51301559)(5.26730798,18.18880166)(5.26730798,22.46431335)
\curveto(5.26730798,26.73881886)(6.80503321,30.44083496)(9.88099548,33.56836664)
\curveto(12.87835428,36.59305128)(16.54097545,38.10490352)(20.86789607,38.10490352)
\curveto(25.24721322,38.10490352)(28.94771777,36.59305128)(31.97128351,33.56836664)
\closepath
\moveto(11.91661781,31.53540946)
\curveto(9.36030935,28.95323508)(8.08268429,25.92855044)(8.08268429,22.46042742)
\curveto(8.08268429,18.99227838)(9.34770469,15.99379774)(11.87727704,13.46404005)
\curveto(14.40736989,10.93522781)(17.41926786,9.66984587)(20.91398593,9.66984587)
\curveto(24.40873002,9.66984587)(27.44638384,10.94685091)(30.02837006,13.5028473)
\curveto(32.47979005,15.87640449)(33.70599885,18.8612757)(33.70599885,22.46043609)
\curveto(33.70599885,26.03240365)(32.45988111,29.06387132)(29.96912035,31.55484779)
\curveto(27.47888877,34.04387261)(24.46067545,35.2898249)(20.91398593,35.2898249)
\curveto(17.36735713,35.28981622)(14.36756067,34.03803504)(11.91661781,31.53540946)
\closepath
\moveto(18.64355083,23.98586307)
\curveto(18.2529278,24.8378532)(17.6682034,25.26289412)(16.8884234,25.26289412)
\curveto(15.50985696,25.26289412)(14.82079928,24.33519776)(14.82079928,22.47983105)
\curveto(14.82079928,20.62446434)(15.50985696,19.69676797)(16.8884234,19.69676797)
\curveto(17.79872627,19.69676797)(18.44897234,20.14896704)(18.83909223,21.05337385)
\lineto(20.74999522,20.03642253)
\curveto(19.83918921,18.41879146)(18.4727503,17.60852304)(16.65066983,17.60852304)
\curveto(15.24539331,17.60852304)(14.11962275,18.03938419)(13.27431239,18.90106311)
\curveto(12.42757935,19.76276806)(12.00542298,20.95050947)(12.00542298,22.46432202)
\curveto(12.00542298,23.95190455)(12.4411556,25.1328716)(13.31315001,26.00812526)
\curveto(14.1851184,26.88245949)(15.27112312,27.31911483)(16.57302061,27.31911483)
\curveto(18.49895724,27.31911483)(19.87802683,26.56125442)(20.71166074,25.04456211)
\lineto(18.64355083,23.98586307)
\closepath
\moveto(27.63366672,23.98586307)
\curveto(27.24254054,24.8378532)(26.66949259,25.26289412)(25.91398502,25.26289412)
\curveto(24.5077022,25.26289412)(23.80413138,24.33519776)(23.80413138,22.47983105)
\curveto(23.80413138,20.62446434)(24.5077022,19.69676797)(25.91398502,19.69676797)
\curveto(26.82574527,19.69676797)(27.46434093,20.14896704)(27.82872232,21.05337385)
\lineto(29.78232327,20.03642253)
\curveto(28.87300067,18.41879146)(27.50849628,17.60852304)(25.68979903,17.60852304)
\curveto(24.28642231,17.60852304)(23.1630894,18.03938419)(22.31831689,18.90106311)
\curveto(21.47496707,19.76276806)(21.0522902,20.95050947)(21.0522902,22.46432202)
\curveto(21.0522902,23.95190455)(21.48127373,25.1328716)(22.33870295,26.00812526)
\curveto(23.195603,26.88245949)(24.28643098,27.31911483)(25.61213246,27.31911483)
\curveto(27.53465984,27.31911483)(28.91182094,26.56125442)(29.74256611,25.04456211)
\lineto(27.63366672,23.98586307)
\closepath
}
}
{
\newrgbcolor{curcolor}{1 1 1}
\pscustom[linestyle=none,fillstyle=solid,fillcolor=curcolor]
{
\newpath
\moveto(74.09876581,26.82228334)
\curveto(74.09876581,20.89185441)(69.29066527,16.08429187)(63.35957268,16.08429187)
\curveto(57.42848009,16.08429187)(52.62037954,20.89185441)(52.62037954,26.82228334)
\curveto(52.62037954,32.75271226)(57.42848009,37.5602748)(63.35957268,37.5602748)
\curveto(69.29066527,37.5602748)(74.09876581,32.75271226)(74.09876581,26.82228334)
\closepath
}
}
{
\newrgbcolor{curcolor}{0 0 0}
\pscustom[linestyle=none,fillstyle=solid,fillcolor=curcolor]
{
\newpath
\moveto(66.46780513,29.92944439)
\curveto(66.46780513,30.34379468)(66.13201653,30.67858184)(65.71858384,30.67858184)
\lineto(60.97580473,30.67858184)
\curveto(60.56237204,30.67858184)(60.22658344,30.34380462)(60.22658344,29.92944439)
\lineto(60.22658344,25.18719597)
\lineto(61.54935538,25.18719597)
\lineto(61.54935538,19.57063755)
\lineto(65.14404935,19.57063755)
\lineto(65.14404935,25.18719597)
\lineto(66.46779519,25.18719597)
\lineto(66.46779519,29.92944439)
\lineto(66.46780513,29.92944439)
\closepath
}
}
{
\newrgbcolor{curcolor}{0 0 0}
\pscustom[linestyle=none,fillstyle=solid,fillcolor=curcolor]
{
\newpath
\moveto(64.96936726,32.9250061)
\curveto(64.96936726,32.02920753)(64.24309777,31.3030193)(63.34719894,31.3030193)
\curveto(62.45130012,31.3030193)(61.72503062,32.02920753)(61.72503062,32.9250061)
\curveto(61.72503062,33.82080468)(62.45130012,34.54699291)(63.34719894,34.54699291)
\curveto(64.24309777,34.54699291)(64.96936726,33.82080468)(64.96936726,32.9250061)
\closepath
}
}
{
\newrgbcolor{curcolor}{0 0 0}
\pscustom[linestyle=none,fillstyle=solid,fillcolor=curcolor]
{
\newpath
\moveto(63.33215838,38.65315797)
\curveto(60.12080541,38.65315797)(57.40101618,37.53334202)(55.17568259,35.29078872)
\curveto(52.89212338,32.97254761)(51.75082576,30.22734654)(51.75082576,27.05905089)
\curveto(51.75082576,23.89075525)(52.89212338,21.16496055)(55.17568259,18.88456831)
\curveto(57.4592418,16.60417607)(60.17855402,15.46397995)(63.33215838,15.46397995)
\curveto(66.52506677,15.46397995)(69.29290531,16.61388422)(71.63372617,18.91368283)
\curveto(73.84061519,21.09704323)(74.94309574,23.81216592)(74.94309574,27.05906083)
\curveto(74.94309574,30.30595574)(73.82120665,33.04921915)(71.57646452,35.29079866)
\curveto(69.33172239,37.53334202)(66.58329239,38.65315797)(63.33215838,38.65315797)
\closepath
\moveto(63.36126622,36.56683935)
\curveto(65.99323504,36.56683935)(68.22826794,35.63915934)(70.06637484,33.7837894)
\curveto(71.92486419,31.94879963)(72.85362191,29.70624633)(72.85362191,27.05905089)
\curveto(72.85362191,24.39244909)(71.94427273,22.17901031)(70.1246104,20.42068214)
\curveto(68.20886934,18.52747326)(65.95441796,17.58037696)(63.36127616,17.58037696)
\curveto(60.76813435,17.58037696)(58.53309152,18.51776511)(56.65665449,20.39156762)
\curveto(54.7787566,22.26634393)(53.84077659,24.48851707)(53.84077659,27.05905089)
\curveto(53.84077659,29.62958472)(54.78894285,31.87116422)(56.68527538,33.7837894)
\curveto(58.50446069,35.63915934)(60.7302713,36.56683935)(63.36126622,36.56683935)
\closepath
}
}
{
\newrgbcolor{curcolor}{0 0 0}
\pscustom[linestyle=none,fillstyle=solid,fillcolor=curcolor]
{
\newpath
\moveto(117.75330429,42.00000611)
\lineto(2.24715569,42.00000611)
\curveto(1.00832842,42.00000611)(-0.00001127,40.99177925)(-0.00001127,39.7535775)
\lineto(-0.00001127,0.50751355)
\curveto(-0.00001127,0.22707018)(0.22756453,0.00000674)(0.50755234,0.00000674)
\lineto(119.49242069,0.00000674)
\curveto(119.77240849,0.00000674)(119.9999843,0.22708011)(119.9999843,0.50751355)
\lineto(119.9999843,39.7535775)
\curveto(119.9999843,40.99177925)(118.99213156,42.00000611)(117.75330429,42.00000611)
\closepath
\moveto(2.24714575,40.98498255)
\lineto(117.75329435,40.98498255)
\curveto(118.43263297,40.98498255)(118.98484715,40.4328401)(118.98484715,39.75356756)
\lineto(118.98484715,12.54797702)
\lineto(36.42776929,12.54797702)
\curveto(33.40178626,7.07697131)(27.57205939,3.36235612)(20.88150966,3.36235612)
\curveto(14.18804814,3.36235612)(8.3597722,7.07405986)(5.33525002,12.54797702)
\lineto(1.01511594,12.54797702)
\lineto(1.01511594,39.75356756)
\curveto(1.01511594,40.4328401)(1.56780713,40.98498255)(2.24714575,40.98498255)
\closepath
}
}
{
\newrgbcolor{curcolor}{1 1 1}
\pscustom[linestyle=none,fillstyle=solid,fillcolor=curcolor]
{
\newpath
\moveto(59.99658281,9.25257115)
\curveto(60.31247592,9.25257115)(60.60022515,9.22443043)(60.86031746,9.16912278)
\curveto(61.12040977,9.11381513)(61.34362288,9.02259626)(61.52995678,8.89547611)
\curveto(61.71531677,8.76932975)(61.85992181,8.60047548)(61.9618241,8.39088075)
\curveto(62.06372638,8.18031223)(62.11516447,7.92121288)(62.11516447,7.61166492)
\curveto(62.11516447,7.27785156)(62.03946819,7.00032958)(61.88710172,6.77714145)
\curveto(61.73570915,6.55492711)(61.51054824,6.37248938)(61.21358666,6.2308219)
\curveto(61.62313366,6.1134099)(61.92884051,5.90768055)(62.13069727,5.61366365)
\curveto(62.33255402,5.31964675)(62.43349234,4.96545322)(62.43349234,4.551093)
\curveto(62.43349234,4.21727963)(62.36846926,3.92811184)(62.23842311,3.68356975)
\curveto(62.10837695,3.43902767)(61.93271626,3.24010495)(61.71338884,3.08581787)
\curveto(61.49308751,2.93055699)(61.24173053,2.81605645)(60.96029181,2.74230629)
\curveto(60.67787918,2.66758233)(60.38866909,2.63071719)(60.09073361,2.63071719)
\lineto(56.87453099,2.63071719)
\lineto(56.87453099,9.25259103)
\lineto(59.99658281,9.25259103)
\lineto(59.99658281,9.25257115)
\closepath
\moveto(59.81073587,6.57432354)
\curveto(60.07034123,6.57432354)(60.28433204,6.6354541)(60.45174436,6.7596628)
\curveto(60.61866972,6.8828977)(60.70212743,7.08280415)(60.70212743,7.36032613)
\curveto(60.70212743,7.51461321)(60.67398356,7.64076951)(60.61866972,7.73974893)
\curveto(60.56238197,7.83872836)(60.48862356,7.91538997)(60.39545661,7.97069762)
\curveto(60.30277662,8.02697907)(60.19602468,8.06482794)(60.07519088,8.08714576)
\curveto(59.95436701,8.10848979)(59.8291705,8.11917173)(59.69912435,8.11917173)
\lineto(58.33412667,8.11917173)
\lineto(58.33412667,6.57432354)
\lineto(59.81073587,6.57432354)
\closepath
\moveto(59.89565444,3.76410668)
\curveto(60.03831167,3.76410668)(60.17418138,3.77769014)(60.30471449,3.80583086)
\curveto(60.43427369,3.83397158)(60.54976096,3.87958102)(60.64875146,3.94459682)
\curveto(60.74774197,4.00961262)(60.82635003,4.09792004)(60.88554955,4.20853534)
\curveto(60.94474908,4.32012444)(60.97386685,4.46180185)(60.97386685,4.63550523)
\curveto(60.97386685,4.9751415)(60.87778813,5.217736)(60.68563067,5.36232486)
\curveto(60.49347321,5.50787758)(60.23969141,5.58066388)(59.92379831,5.58066388)
\lineto(58.3341366,5.58066388)
\lineto(58.3341366,3.76411661)
\lineto(59.89565444,3.76411661)
\lineto(59.89565444,3.76410668)
\closepath
}
}
{
\newrgbcolor{curcolor}{1 1 1}
\pscustom[linestyle=none,fillstyle=solid,fillcolor=curcolor]
{
\newpath
\moveto(62.69066294,9.25257115)
\lineto(64.32399137,9.25257115)
\lineto(65.874836,6.63739175)
\lineto(67.41597139,9.25257115)
\lineto(69.03960052,9.25257115)
\lineto(66.58135452,5.17211662)
\lineto(66.58135452,2.63068738)
\lineto(65.12173897,2.63068738)
\lineto(65.12173897,5.2089917)
\lineto(62.69066294,9.25257115)
\closepath
}
}
{
\newrgbcolor{curcolor}{1 1 1}
\pscustom[linestyle=none,fillstyle=solid,fillcolor=curcolor]
{
\newpath
\moveto(85.93291015,9.25257115)
\lineto(88.69977478,4.81017247)
\lineto(88.71529764,4.81017247)
\lineto(88.71529764,9.25257115)
\lineto(90.08174624,9.25257115)
\lineto(90.08174624,2.63069732)
\lineto(88.62504246,2.63069732)
\lineto(85.86885104,7.06436164)
\lineto(85.85041641,7.06436164)
\lineto(85.85041641,2.63069732)
\lineto(84.48396781,2.63069732)
\lineto(84.48396781,9.25257115)
\lineto(85.93291015,9.25257115)
\closepath
}
}
{
\newrgbcolor{curcolor}{1 1 1}
\pscustom[linestyle=none,fillstyle=solid,fillcolor=curcolor]
{
\newpath
\moveto(94.21797124,9.25257115)
\curveto(94.64595288,9.25257115)(95.04288881,9.1846439)(95.41167093,9.04878939)
\curveto(95.78045304,8.91196109)(96.09878091,8.70818926)(96.36857252,8.43648025)
\curveto(96.63740016,8.16477123)(96.84799225,7.8241711)(96.99939476,7.41660758)
\curveto(97.15176123,7.0080802)(97.22745751,6.52871411)(97.22745751,5.97849938)
\curveto(97.22745751,5.49622184)(97.16534621,5.05081938)(97.04209752,4.64325585)
\curveto(96.91787492,4.23472847)(96.73056711,3.88248253)(96.48018404,3.58554425)
\curveto(96.22882706,3.28957976)(95.91633269,3.05571962)(95.54171708,2.88590148)
\curveto(95.16711141,2.71510955)(94.72650875,2.63068738)(94.21797124,2.63068738)
\lineto(91.35793966,2.63068738)
\lineto(91.35793966,9.25256122)
\lineto(94.21797124,9.25256122)
\lineto(94.21797124,9.25257115)
\closepath
\moveto(94.11606896,3.8572632)
\curveto(94.32666105,3.8572632)(94.53046561,3.89122683)(94.72844663,3.95915408)
\curveto(94.92642764,4.02708133)(95.1030523,4.13964423)(95.25736658,4.29684276)
\curveto(95.41167093,4.45501508)(95.53589353,4.65976071)(95.62906048,4.91303715)
\curveto(95.72125352,5.16630366)(95.76784196,5.47488775)(95.76784196,5.83975329)
\curveto(95.76784196,6.17356666)(95.73581241,6.47438026)(95.67078933,6.74317782)
\curveto(95.60576625,7.01197538)(95.49901432,7.24195028)(95.35052359,7.43312237)
\curveto(95.20203287,7.62429446)(95.00599966,7.77081105)(94.76144015,7.87367572)
\curveto(94.51688063,7.97459281)(94.21505946,8.02602515)(93.85694061,8.02602515)
\lineto(92.81754528,8.02602515)
\lineto(92.81754528,3.85727314)
\lineto(94.11606896,3.85727314)
\lineto(94.11606896,3.8572632)
\closepath
}
}
{
\newrgbcolor{curcolor}{1 1 1}
\pscustom[linestyle=none,fillstyle=solid,fillcolor=curcolor]
{
\newpath
\moveto(102.49426877,27.01927231)
\curveto(102.49817554,21.17467248)(97.76218954,16.43242846)(91.91594809,16.42757815)
\curveto(86.07069419,16.42465927)(81.32791948,21.15816528)(81.32306863,27.00471517)
\lineto(81.32306863,27.01927231)
\curveto(81.31919912,32.86484096)(86.05514786,37.60613479)(91.90138931,37.6109851)
\curveto(97.74763077,37.61583541)(102.48941792,32.88036072)(102.49426877,27.03479207)
\lineto(102.49426877,27.01927231)
\closepath
}
}
{
\newrgbcolor{curcolor}{0 0 0}
\pscustom[linestyle=none,fillstyle=solid,fillcolor=curcolor]
{
\newpath
\moveto(91.74318689,38.61433496)
\curveto(88.52990933,38.61433496)(85.81157414,37.49454038)(83.58625586,35.25293906)
\curveto(81.30270871,32.93372656)(80.16140407,30.18950634)(80.16140407,27.01927231)
\curveto(80.16140407,23.8500009)(81.30270249,21.12615074)(83.58625586,18.84479054)
\curveto(85.86980301,16.56439294)(88.58813821,15.42419104)(91.74318689,15.42419104)
\curveto(94.93516044,15.42419104)(97.70300416,16.57506859)(100.04474292,18.87390482)
\curveto(102.25069505,21.05822776)(103.35416179,23.77335853)(103.35416179,27.01927231)
\curveto(103.35416179,30.26714236)(102.23128542,33.01039376)(99.98651404,35.25293906)
\curveto(97.74182341,37.49454038)(94.99338931,38.61433496)(91.74318689,38.61433496)
\closepath
\moveto(91.77230443,36.52898072)
\curveto(94.40329321,36.52898072)(96.63835667,35.60034181)(98.47742028,33.74400795)
\curveto(100.33494323,31.90901919)(101.26368607,29.66840532)(101.26368607,27.0192661)
\curveto(101.26368607,24.35265706)(100.35435285,22.14018866)(98.5346616,20.3818609)
\curveto(96.61894704,18.48768096)(94.36447396,17.54156602)(91.77230443,17.54156602)
\curveto(89.17820326,17.54156602)(86.94317707,18.47797413)(85.06623829,20.35371544)
\curveto(83.18930572,22.2275191)(82.25085496,24.44969434)(82.25085496,27.0192661)
\curveto(82.25085496,29.59077551)(83.19900742,31.83138316)(85.09534962,33.74400795)
\curveto(86.91504087,35.60034181)(89.14032809,36.52898072)(91.77230443,36.52898072)
\closepath
}
}
{
\newrgbcolor{curcolor}{0 0 0}
\pscustom[linestyle=none,fillstyle=solid,fillcolor=curcolor]
{
\newpath
\moveto(96.13954481,29.76739887)
\lineto(87.70886663,29.76739887)
\lineto(87.70886663,27.77131271)
\lineto(96.13954481,27.77131271)
\lineto(96.13954481,29.76739887)
\closepath
\moveto(96.13954481,26.04113041)
\lineto(87.70886663,26.04113041)
\lineto(87.70886663,24.04506289)
\lineto(96.13954481,24.04506289)
\lineto(96.13954481,26.04113041)
\closepath
}
}
\end{pspicture}
 \\ {\tt cc-by-nd}

        \vspace{3pt}%LaTeX with PSTricks extensions
%%Creator: inkscape 0.91
%%Please note this file requires PSTricks extensions
\psset{xunit=.5pt,yunit=.5pt,runit=.5pt}
\begin{pspicture}(120,42)
{
\newrgbcolor{curcolor}{0.66666669 0.69803923 0.67058825}
\pscustom[linestyle=none,fillstyle=solid,fillcolor=curcolor]
{
\newpath
\moveto(3.40785387,41.52353382)
\lineto(116.76241589,41.32169969)
\curveto(118.34625388,41.32169969)(119.76122865,41.55702048)(119.76122865,38.16165214)
\lineto(119.62244718,0.83258309)
\lineto(0.54733563,0.83258309)
\lineto(0.54733563,38.30041807)
\curveto(0.54733563,39.97432373)(0.70941133,41.52353382)(3.40785387,41.52353382)
\closepath
}
}
{
\newrgbcolor{curcolor}{1 1 1}
\pscustom[linestyle=none,fillstyle=solid,fillcolor=curcolor]
{
\newpath
\moveto(34.52222482,22.42449543)
\curveto(34.52706542,14.89529965)(28.42561219,8.7882238)(20.89557306,8.78290667)
\curveto(13.36555996,8.778535)(7.2568198,14.87883649)(7.25245631,22.40798024)
\lineto(7.25245631,22.42449543)
\curveto(7.24808415,29.95464536)(13.34904291,36.06172121)(20.87908204,36.06611023)
\curveto(28.41007541,36.07098498)(34.51786133,29.97015438)(34.52222482,22.44101063)
\lineto(34.52222482,22.42449543)
\closepath
}
}
{
\newrgbcolor{curcolor}{0 0 0}
\pscustom[linestyle=none,fillstyle=solid,fillcolor=curcolor]
{
\newpath
\moveto(31.97128639,33.52952664)
\curveto(34.99485183,30.50582276)(36.50687311,26.80382476)(36.50687311,22.42450411)
\curveto(36.50687311,18.04469772)(35.02103274,14.38246977)(32.04940405,11.43690082)
\curveto(28.89580155,8.33457627)(25.16863934,6.78391709)(20.86789139,6.78391709)
\curveto(16.61911489,6.78391709)(12.95647674,8.32146993)(9.88101793,11.39806755)
\curveto(6.80502994,14.47368501)(5.26730487,18.14899328)(5.26730487,22.42450411)
\curveto(5.26730487,26.69953787)(6.80502994,30.40105014)(9.88101793,33.52952664)
\curveto(12.87837643,36.55370759)(16.54097121,38.06508246)(20.86789139,38.06508246)
\curveto(25.2472428,38.06508246)(28.94772095,36.55370759)(31.97128639,33.52952664)
\closepath
\moveto(11.9166227,31.49561574)
\curveto(9.36034053,28.91391895)(8.08268957,25.88973801)(8.08268957,22.42063553)
\curveto(8.08268957,18.95248718)(9.34773587,15.95403317)(11.87730796,13.4251694)
\curveto(14.40737453,10.89541222)(17.4192722,9.63052495)(20.91401594,9.63052495)
\curveto(24.40873366,9.63052495)(27.44638717,10.90803281)(30.02837314,13.46355163)
\curveto(32.4798189,15.83660525)(33.70602758,18.82150188)(33.70602758,22.42063553)
\curveto(33.70602758,25.99260237)(32.45990996,29.02404341)(29.96914946,31.51504539)
\curveto(27.4788921,34.00504987)(24.46067909,35.25049881)(20.91401594,35.25049881)
\curveto(17.3673875,35.25049881)(14.36759135,33.99869186)(11.9166227,31.49561574)
\closepath
\moveto(18.64358107,23.94657396)
\curveto(18.25293205,24.7980348)(17.66823374,25.22357873)(16.8884278,25.22357873)
\curveto(15.50986149,25.22357873)(14.82082991,24.29588255)(14.82082991,22.44054224)
\curveto(14.82082991,20.58467282)(15.50986149,19.65697664)(16.8884278,19.65697664)
\curveto(17.7987566,19.65697664)(18.44900261,20.10914959)(18.83912245,21.01455372)
\lineto(20.74999922,19.9970735)
\curveto(19.8391933,18.37896569)(18.47278056,17.56922654)(16.65067425,17.56922654)
\curveto(15.24542389,17.56922654)(14.11965344,18.00006158)(13.27434317,18.86126326)
\curveto(12.42761022,19.72294202)(12.00542786,20.91068318)(12.00542786,22.42449543)
\curveto(12.00542786,23.91258075)(12.44116044,25.09354757)(13.31315477,25.96835)
\curveto(14.18514909,26.84262333)(15.27112768,27.28029344)(16.57302504,27.28029344)
\curveto(18.49896147,27.28029344)(19.87803092,26.52142701)(20.71166475,25.00523809)
\lineto(18.64358107,23.94657396)
\closepath
\moveto(27.63369605,23.94657396)
\curveto(27.24256992,24.7980348)(26.669496,25.22357873)(25.9139885,25.22357873)
\curveto(24.50773185,25.22357873)(23.80413508,24.29588255)(23.80413508,22.44054224)
\curveto(23.80413508,20.58467282)(24.50773185,19.65697664)(25.9139885,19.65697664)
\curveto(26.82574867,19.65697664)(27.46434426,20.10914959)(27.82875164,21.01455372)
\lineto(29.78235239,19.9970735)
\curveto(28.87300386,18.37896569)(27.50849093,17.56922654)(25.68980254,17.56922654)
\curveto(24.28645198,17.56922654)(23.16311919,18.00006158)(22.31832074,18.86126326)
\curveto(21.474971,19.72294202)(21.0523202,20.91068318)(21.0523202,22.42449543)
\curveto(21.0523202,23.91258075)(21.48127767,25.09354757)(22.3387068,25.96835)
\curveto(23.19563279,26.84262333)(24.28646066,27.28029344)(25.612162,27.28029344)
\curveto(27.53468919,27.28029344)(28.91182413,26.52142701)(29.74256921,25.00523809)
\lineto(27.63369605,23.94657396)
\closepath
}
}
{
\newrgbcolor{curcolor}{0 0 0}
\pscustom[linestyle=none,fillstyle=solid,fillcolor=curcolor]
{
\newpath
\moveto(117.75332456,41.9999983)
\lineto(2.24668075,41.9999983)
\curveto(1.00785361,41.9999983)(0.00000097,40.99177164)(0.00000097,39.75308324)
\lineto(0.00000097,0.50750415)
\curveto(0.00000097,0.22754773)(0.22709974,-0.00000256)(0.50707758,-0.00000256)
\lineto(119.49244079,-0.00000256)
\curveto(119.77242856,-0.00000256)(120.00000434,0.22754773)(120.00000434,0.50750415)
\lineto(120.00000434,39.7530733)
\curveto(120.00000434,40.99177164)(118.99215171,41.9999983)(117.75332456,41.9999983)
\closepath
\moveto(2.24668075,40.98497494)
\lineto(117.75332456,40.98497494)
\curveto(118.43266311,40.98497494)(118.98487723,40.43234571)(118.98487723,39.7530733)
\lineto(118.98487723,12.50818861)
\lineto(36.42778781,12.50818861)
\curveto(33.40180508,7.0376709)(27.5720788,3.32305646)(20.88152974,3.32305646)
\curveto(14.1880689,3.32305646)(8.35979354,7.03427255)(5.33527167,12.50818861)
\lineto(1.01417405,12.50818861)
\lineto(1.01417405,39.7530733)
\curveto(1.01415418,40.43234571)(1.5673422,40.98497494)(2.24668075,40.98497494)
\closepath
}
}
{
\newrgbcolor{curcolor}{1 1 1}
\pscustom[linestyle=none,fillstyle=solid,fillcolor=curcolor]
{
\newpath
\moveto(86.2638626,4.26772802)
\curveto(86.34344456,4.11295407)(86.45019648,3.9877716)(86.5831544,3.89170364)
\curveto(86.71611232,3.79612259)(86.87139055,3.724797)(87.04996301,3.67870067)
\curveto(87.22949943,3.63211745)(87.41486934,3.60883578)(87.60605287,3.60883578)
\curveto(87.73512511,3.60883578)(87.87390659,3.61950779)(88.0223973,3.6413387)
\curveto(88.1699141,3.66268272)(88.30869558,3.7048938)(88.43874172,3.76652117)
\curveto(88.56781396,3.82813861)(88.67651369,3.91353456)(88.76191921,4.02173522)
\curveto(88.84926259,4.12993588)(88.89196535,4.26724112)(88.89196535,4.43463467)
\curveto(88.89196535,4.61367402)(88.8347037,4.75873982)(88.72019035,4.87033883)
\curveto(88.60664097,4.9819279)(88.45622232,5.07412055)(88.27085241,5.14883455)
\curveto(88.0864564,5.22258469)(87.87586433,5.28760048)(87.64197805,5.34339502)
\curveto(87.40711785,5.39918955)(87.16934588,5.46080699)(86.92867207,5.52922113)
\curveto(86.6812008,5.59083857)(86.44051705,5.66652637)(86.20565686,5.75629446)
\curveto(85.97177057,5.84556572)(85.76116857,5.96201384)(85.57580859,6.10417812)
\curveto(85.39044862,6.2463424)(85.24099394,6.42392106)(85.12647065,6.63740099)
\curveto(85.01292127,6.85088093)(84.95565962,7.10900643)(84.95565962,7.41176756)
\curveto(84.95565962,7.75236762)(85.02844412,8.04736819)(85.17304915,8.29772319)
\curveto(85.31861815,8.5480782)(85.50883771,8.75670903)(85.7436979,8.92410257)
\curveto(85.97758419,9.09100922)(86.24350003,9.214731)(86.54047152,9.29527784)
\curveto(86.8364691,9.37533779)(87.13344059,9.41560127)(87.42943817,9.41560127)
\curveto(87.77589993,9.41560127)(88.10781275,9.37678854)(88.42613064,9.29915315)
\curveto(88.74348457,9.22200466)(89.02686114,9.09634523)(89.27336844,8.92312878)
\curveto(89.52083971,8.74991233)(89.71688283,8.52867184)(89.86245183,8.25938743)
\curveto(90.00705686,7.99010302)(90.07984136,7.66357332)(90.07984136,7.28027529)
\lineto(88.66680445,7.28027529)
\curveto(88.65418344,7.4782341)(88.61245459,7.6422293)(88.54258187,7.77177398)
\curveto(88.47173524,7.90180556)(88.3775944,8.00369642)(88.26114317,8.07841042)
\curveto(88.14371805,8.15216056)(88.00978622,8.20504365)(87.85935764,8.23560892)
\curveto(87.70796508,8.26666109)(87.54395152,8.28219214)(87.36537906,8.28219214)
\curveto(87.2489179,8.28219214)(87.13149278,8.26957254)(87.01503162,8.24531707)
\curveto(86.89760649,8.2200878)(86.79181853,8.17690293)(86.69671372,8.11528549)
\curveto(86.600635,8.05318116)(86.52202695,7.97603266)(86.46088956,7.88336305)
\curveto(86.39975217,7.79020655)(86.36869652,7.67279457)(86.36869652,7.53063029)
\curveto(86.36869652,7.40059871)(86.39295471,7.2953095)(86.44245493,7.2152595)
\curveto(86.49194521,7.13472259)(86.58997173,7.06048555)(86.73457676,6.99255831)
\curveto(86.87918179,6.92463107)(87.08007456,6.85621693)(87.33628116,6.78828969)
\curveto(87.59248777,6.72036245)(87.92731236,6.63351575)(88.34170897,6.52871344)
\curveto(88.46495766,6.50397107)(88.63576868,6.45884854)(88.85509608,6.39383275)
\curveto(89.07442348,6.32881696)(89.29181301,6.22547534)(89.50823857,6.08331106)
\curveto(89.7246542,5.94066982)(89.91098808,5.75047156)(90.06917809,5.51272622)
\curveto(90.2263942,5.27449398)(90.30500225,4.9697952)(90.30500225,4.59862987)
\curveto(90.30500225,4.29538184)(90.24580273,4.01397466)(90.12837761,3.75439841)
\curveto(90.01095248,3.49433525)(89.83625577,3.27018331)(89.60431729,3.0819227)
\curveto(89.37236888,2.89269824)(89.08510663,2.74617175)(88.74252061,2.64088254)
\curveto(88.39897062,2.53559333)(88.00107076,2.48319714)(87.55075892,2.48319714)
\curveto(87.18585258,2.48319714)(86.83162939,2.52831967)(86.48806946,2.61807783)
\curveto(86.14548344,2.70783598)(85.84172443,2.84902647)(85.57872037,3.04116239)
\curveto(85.31669021,3.2332983)(85.10803602,3.4783173)(84.95275778,3.77525553)
\curveto(84.79845345,4.07268065)(84.72469505,4.42492652)(84.7305186,4.83345382)
\lineto(86.14355551,4.83345382)
\curveto(86.14352569,4.61075263)(86.18331667,4.42201507)(86.2638626,4.26772802)
\closepath
}
}
{
\newrgbcolor{curcolor}{1 1 1}
\pscustom[linestyle=none,fillstyle=solid,fillcolor=curcolor]
{
\newpath
\moveto(94.46935083,9.25256993)
\lineto(96.94700512,2.63069743)
\lineto(95.43401376,2.63069743)
\lineto(94.93323773,4.10568041)
\lineto(92.45558344,4.10568041)
\lineto(91.93637277,2.63069743)
\lineto(90.47093381,2.63069743)
\lineto(92.97576801,9.25256993)
\lineto(94.46935083,9.25256993)
\closepath
\moveto(94.55280854,5.19249638)
\lineto(93.71819175,7.62038845)
\lineto(93.69975712,7.62038845)
\lineto(92.83602256,5.19249638)
\lineto(94.55280854,5.19249638)
\closepath
}
}
{
\newrgbcolor{curcolor}{1 1 1}
\pscustom[linestyle=none,fillstyle=solid,fillcolor=curcolor]
{
\newpath
\moveto(59.99659896,9.25256993)
\curveto(60.31200508,9.25256993)(60.60024123,9.22442921)(60.86033352,9.16912157)
\curveto(61.1204258,9.11332704)(61.34363888,9.02210819)(61.52997277,8.89547496)
\curveto(61.71533274,8.76884173)(61.85993777,8.60047439)(61.96184004,8.3903928)
\curveto(62.06374232,8.18031122)(62.11518041,7.92072502)(62.11518041,7.61166403)
\curveto(62.11518041,7.27785073)(62.03948413,7.00032881)(61.88711767,6.77761768)
\curveto(61.73572512,6.5549165)(61.51056423,6.37296569)(61.21360268,6.23081135)
\curveto(61.62314964,6.11339937)(61.92885646,5.90767006)(62.1307132,5.61365322)
\curveto(62.33256993,5.31963638)(62.43350824,4.96544292)(62.43350824,4.55108279)
\curveto(62.43350824,4.21726949)(62.36848517,3.92810175)(62.23843903,3.68355972)
\curveto(62.10839288,3.43950457)(61.93273221,3.240095)(61.71340481,3.08532105)
\curveto(61.4931035,2.93054711)(61.24174655,2.81603665)(60.96030785,2.74229644)
\curveto(60.67789526,2.6675725)(60.3886852,2.63070737)(60.09074974,2.63070737)
\lineto(56.87454745,2.63070737)
\lineto(56.87454745,9.25257987)
\lineto(59.99659896,9.25257987)
\lineto(59.99659896,9.25256993)
\closepath
\moveto(59.81026508,6.57432286)
\curveto(60.07035736,6.57432286)(60.28386121,6.6359403)(60.45176046,6.75918512)
\curveto(60.6186858,6.88290689)(60.7021435,7.08280337)(60.7021435,7.36033523)
\curveto(60.7021435,7.51462228)(60.67399963,7.64125551)(60.6186858,7.73975795)
\curveto(60.56239806,7.83873736)(60.48863966,7.91539896)(60.39547272,7.97119349)
\curveto(60.30230578,8.02650113)(60.19555385,8.0653238)(60.07520701,8.08666781)
\curveto(59.9538962,8.10849873)(59.82869971,8.11917074)(59.69865357,8.11917074)
\lineto(58.33414298,8.11917074)
\lineto(58.33414298,6.57432286)
\lineto(59.81026508,6.57432286)
\closepath
\moveto(59.89566065,3.76409662)
\curveto(60.03831787,3.76409662)(60.17418757,3.77768008)(60.30423371,3.8058208)
\curveto(60.43427985,3.83347462)(60.54976711,3.88005784)(60.64875761,3.94458673)
\curveto(60.7477481,4.00960252)(60.82635615,4.09790992)(60.88555567,4.20901209)
\curveto(60.94475519,4.32011427)(60.97387296,4.46227855)(60.97387296,4.635495)
\curveto(60.97387296,4.9751312)(60.87779424,5.21772565)(60.68563681,5.36279144)
\curveto(60.49347937,5.50785724)(60.23921064,5.58064352)(59.92380452,5.58064352)
\lineto(58.33414298,5.58064352)
\lineto(58.33414298,3.76409662)
\lineto(59.89566065,3.76409662)
\closepath
}
}
{
\newrgbcolor{curcolor}{1 1 1}
\pscustom[linestyle=none,fillstyle=solid,fillcolor=curcolor]
{
\newpath
\moveto(62.69066887,9.25256993)
\lineto(64.32399714,9.25256993)
\lineto(65.87484161,6.63739106)
\lineto(67.41597685,9.25256993)
\lineto(69.03960582,9.25256993)
\lineto(66.58136006,5.17211622)
\lineto(66.58136006,2.63068749)
\lineto(65.12174466,2.63068749)
\lineto(65.12174466,5.20899129)
\lineto(62.69066887,9.25256993)
\closepath
}
}
{
\newrgbcolor{curcolor}{1 1 1}
\pscustom[linestyle=none,fillstyle=solid,fillcolor=curcolor]
{
\newpath
\moveto(102.40308729,27.01924657)
\curveto(102.40697543,21.17415729)(97.67195262,16.43191423)(91.82573039,16.42755454)
\curveto(85.98047708,16.42315759)(81.23770285,21.15815934)(81.23285199,27.00370819)
\lineto(81.23285199,27.01924657)
\curveto(81.22896386,32.86479541)(85.96494938,37.60608829)(91.8102089,37.6109386)
\curveto(97.65643113,37.61529829)(102.39919916,32.87984317)(102.40308729,27.03474769)
\lineto(102.40308729,27.01924657)
\closepath
}
}
{
\newrgbcolor{curcolor}{0 0 0}
\pscustom[linestyle=none,fillstyle=solid,fillcolor=curcolor]
{
\newpath
\moveto(91.7422659,38.61383489)
\curveto(88.52993275,38.61383489)(85.81160404,37.49305309)(83.58627977,35.25194905)
\curveto(81.30270801,32.93271838)(80.16140969,30.18849871)(80.16140969,27.01924657)
\curveto(80.16140969,23.84999442)(81.30270801,21.12464811)(83.58627977,18.84476644)
\curveto(85.86886398,16.56435067)(88.58816162,15.42416763)(91.7422659,15.42416763)
\curveto(94.93517079,15.42416763)(97.70202667,16.5740575)(100.04479002,18.87388072)
\curveto(102.24975437,21.05721576)(103.35318373,23.77282419)(103.35318373,27.01924657)
\curveto(103.35318373,30.26517833)(102.23034474,33.01038545)(99.98559222,35.25194905)
\curveto(97.74086455,37.49305309)(94.99339966,38.61383489)(91.7422659,38.61383489)
\closepath
\moveto(91.77138344,36.52704027)
\curveto(94.40334088,36.52704027)(96.63838548,35.59984235)(98.47746754,33.74445905)
\curveto(100.33400274,31.90848321)(101.26277033,29.66692582)(101.26277033,27.01924657)
\curveto(101.26277033,24.35261944)(100.3543875,22.13967328)(98.53471506,20.38087388)
\curveto(96.61898207,18.48716631)(94.36452784,17.54056094)(91.77138965,17.54056094)
\curveto(89.17823283,17.54056094)(86.94320686,18.47795631)(85.06627447,20.3517596)
\curveto(83.18936073,22.2270037)(82.25089143,24.44870029)(82.25089143,27.01924036)
\curveto(82.25089143,29.58928981)(83.19906243,31.83133782)(85.09538581,33.74445284)
\curveto(86.9140831,35.59984235)(89.13940737,36.52704027)(91.77138344,36.52704027)
\closepath
}
}
{
\newrgbcolor{curcolor}{0 0 0}
\pscustom[linestyle=none,fillstyle=solid,fillcolor=curcolor]
{
\newpath
\moveto(86.60254777,28.65580077)
\curveto(87.06450924,31.57324668)(89.11903503,33.13263065)(91.69277601,33.13263065)
\curveto(95.39518198,33.13263065)(97.65059893,30.44663333)(97.65059893,26.86493095)
\curveto(97.65059893,23.37011184)(95.25056941,20.65500024)(91.63454714,20.65500024)
\curveto(89.14717743,20.65500024)(86.92087181,22.18576054)(86.51520136,25.18910228)
\lineto(89.43639014,25.18910228)
\curveto(89.52373034,23.62968105)(90.53595,23.08093151)(91.9819701,23.08093151)
\curveto(93.62987512,23.08093151)(94.70128637,24.61172908)(94.70128637,26.95180801)
\curveto(94.70128637,29.40635673)(93.7754442,30.70620254)(92.03923004,30.70620254)
\curveto(90.76691521,30.70620254)(89.6683367,30.24379597)(89.43639014,28.65580077)
\lineto(90.28652546,28.66016046)
\lineto(87.98646328,26.36132469)
\lineto(85.68737002,28.66016046)
\lineto(86.60254777,28.65580077)
\closepath
}
}
{
\newrgbcolor{curcolor}{1 1 1}
\pscustom[linestyle=none,fillstyle=solid,fillcolor=curcolor]
{
\newpath
\moveto(74.09879048,26.78297554)
\curveto(74.09879048,20.8525478)(69.29069042,16.04498624)(63.35959842,16.04498624)
\curveto(57.42850643,16.04498624)(52.62040637,20.8525478)(52.62040637,26.78297554)
\curveto(52.62040637,32.71340327)(57.42850643,37.52096484)(63.35959842,37.52096484)
\curveto(69.29069042,37.52096484)(74.09879048,32.71340327)(74.09879048,26.78297554)
\closepath
}
}
{
\newrgbcolor{curcolor}{0 0 0}
\pscustom[linestyle=none,fillstyle=solid,fillcolor=curcolor]
{
\newpath
\moveto(66.46783056,29.89012945)
\curveto(66.46783056,30.30399275)(66.132042,30.63877985)(65.71860935,30.63877985)
\lineto(60.97583071,30.63877985)
\curveto(60.56239806,30.63877985)(60.2266095,30.30400269)(60.2266095,29.89012945)
\lineto(60.2266095,25.14739508)
\lineto(61.5493813,25.14739508)
\lineto(61.5493813,19.53083779)
\lineto(65.14407491,19.53083779)
\lineto(65.14407491,25.14739508)
\lineto(66.46782062,25.14739508)
\lineto(66.46782062,29.89012945)
\lineto(66.46783056,29.89012945)
\closepath
}
}
{
\newrgbcolor{curcolor}{0 0 0}
\pscustom[linestyle=none,fillstyle=solid,fillcolor=curcolor]
{
\newpath
\moveto(64.96939285,32.88569708)
\curveto(64.96939285,31.98989868)(64.24312342,31.2637106)(63.34722469,31.2637106)
\curveto(62.45132595,31.2637106)(61.72505653,31.98989868)(61.72505653,32.88569708)
\curveto(61.72505653,33.78149547)(62.45132595,34.50768356)(63.34722469,34.50768356)
\curveto(64.24312342,34.50768356)(64.96939285,33.78149547)(64.96939285,32.88569708)
\closepath
}
}
{
\newrgbcolor{curcolor}{0 0 0}
\pscustom[linestyle=none,fillstyle=solid,fillcolor=curcolor]
{
\newpath
\moveto(63.33218413,38.61385121)
\curveto(60.12083148,38.61385121)(57.40104253,37.49354858)(55.17570916,35.25148263)
\curveto(52.89215017,32.93324198)(51.75085267,30.18852837)(51.75085267,27.01974646)
\curveto(51.75085267,23.85096455)(52.89215017,21.1251704)(55.17570916,18.84477862)
\curveto(57.45926814,16.56487374)(60.17858008,15.42467785)(63.33218413,15.42467785)
\curveto(66.5250922,15.42467785)(69.29293045,16.57409499)(71.63375108,18.87438003)
\curveto(73.84063988,21.0572531)(74.94312031,23.77286214)(74.94312031,27.0197564)
\curveto(74.94312031,30.26665066)(73.82123134,33.01040041)(71.57648944,35.25149257)
\curveto(69.33174753,37.49354858)(66.58331781,38.61385121)(63.33218413,38.61385121)
\closepath
\moveto(63.36129196,36.527533)
\curveto(65.99326052,36.527533)(68.22829319,35.59985319)(70.06639991,33.74448362)
\curveto(71.92488907,31.90900731)(72.8536467,29.66694136)(72.8536467,27.01974646)
\curveto(72.8536467,24.3531452)(71.94429761,22.13970686)(70.12463546,20.38137905)
\curveto(68.20889459,18.48768365)(65.95444345,17.54107443)(63.3613019,17.54107443)
\curveto(60.76816036,17.54107443)(58.53311775,18.4779755)(56.65668091,20.35275143)
\curveto(54.77878321,22.22704046)(53.84080329,24.44921315)(53.84080329,27.01974646)
\curveto(53.84080329,29.58980281)(54.78896946,31.83138186)(56.68530179,33.74448362)
\curveto(58.50448693,35.59985319)(60.73029731,36.527533)(63.36129196,36.527533)
\closepath
}
}
\end{pspicture}
 \\ {\tt cc-by-sa}

        \vspace{3pt}%LaTeX with PSTricks extensions
%%Creator: inkscape 0.91
%%Please note this file requires PSTricks extensions
\psset{xunit=.5pt,yunit=.5pt,runit=.5pt}
\begin{pspicture}(120,42)
{
\newrgbcolor{curcolor}{0.66666669 0.69803923 0.67058825}
\pscustom[linestyle=none,fillstyle=solid,fillcolor=curcolor]
{
\newpath
\moveto(3.13998263,41.49249256)
\lineto(116.49364354,41.29113536)
\curveto(118.07746894,41.29113536)(119.49243247,41.52596929)(119.49243247,38.13061021)
\lineto(119.3536521,0.80153365)
\lineto(0.27948713,0.80153365)
\lineto(0.27948713,38.26937617)
\curveto(0.27948713,39.94375912)(0.44156154,41.49249256)(3.13998263,41.49249256)
\closepath
}
}
{
\newrgbcolor{curcolor}{0 0 0}
\pscustom[linestyle=none,fillstyle=solid,fillcolor=curcolor]
{
\newpath
\moveto(117.75236611,41.99999937)
\lineto(2.24714734,41.99999937)
\curveto(1.00833004,41.99999937)(-0.00000153,40.99225941)(-0.00000153,39.75357076)
\lineto(-0.00000153,0.50701991)
\curveto(-0.00000153,0.22706344)(0.22757244,-0)(0.50755799,-0)
\lineto(119.4924424,-0)
\curveto(119.77242795,-0)(120.00000192,0.22707338)(120.00000192,0.50701991)
\lineto(120.00000192,39.75357076)
\curveto(119.99999199,40.99225941)(118.99167035,41.99999937)(117.75236611,41.99999937)
\closepath
\moveto(2.2471374,40.98498575)
\lineto(117.75235617,40.98498575)
\curveto(118.43217627,40.98498575)(118.98486301,40.4328433)(118.98486301,39.75357076)
\lineto(118.98486301,12.47762841)
\lineto(36.15962426,12.47762841)
\curveto(33.13366559,7.0066227)(27.30398565,3.29200751)(20.61348977,3.29200751)
\curveto(13.92008215,3.29200751)(8.09185312,7.00322435)(5.06735529,12.47762841)
\lineto(1.01510757,12.47762841)
\lineto(1.01510757,39.75357076)
\curveto(1.01511751,40.43283337)(1.56780425,40.98498575)(2.2471374,40.98498575)
\closepath
}
}
{
\newrgbcolor{curcolor}{1 1 1}
\pscustom[linestyle=none,fillstyle=solid,fillcolor=curcolor]
{
\newpath
\moveto(73.80897511,9.25257435)
\curveto(74.12437872,9.25257435)(74.41261258,9.22443363)(74.6736767,9.16912598)
\curveto(74.93376691,9.11381833)(75.15600432,9.02259946)(75.34233672,8.89547931)
\curveto(75.52769523,8.76933295)(75.67229911,8.60047868)(75.77420057,8.39088395)
\curveto(75.87610203,8.18031543)(75.92753971,7.92121608)(75.92753971,7.61166812)
\curveto(75.92753971,7.27785476)(75.85184404,7.00033278)(75.69947879,6.77810851)
\curveto(75.54808744,6.55492038)(75.32292834,6.37345643)(75.02596915,6.23081516)
\curveto(75.43551286,6.11340316)(75.74121725,5.90767381)(75.94307238,5.61365691)
\curveto(76.14492751,5.31964001)(76.24586502,4.96544648)(76.24586502,4.55108626)
\curveto(76.24586502,4.2172729)(76.18084246,3.92810511)(76.05079735,3.68356302)
\curveto(75.92075224,3.43999472)(75.74509296,3.24009821)(75.52576731,3.08581113)
\curveto(75.30546775,2.93055026)(75.0541128,2.81604971)(74.77267634,2.74229955)
\curveto(74.49026599,2.6675756)(74.20105823,2.63071046)(73.90312514,2.63071046)
\lineto(70.68692854,2.63071046)
\lineto(70.68692854,9.25258429)
\lineto(73.80897511,9.25258429)
\lineto(73.80897511,9.25257435)
\closepath
\moveto(73.62264271,6.57431681)
\curveto(73.88273292,6.57431681)(74.09623507,6.63642116)(74.26413299,6.75965606)
\curveto(74.431057,6.88289097)(74.51451404,7.08279742)(74.51451404,7.3603194)
\curveto(74.51451404,7.51460648)(74.4863704,7.64172663)(74.431057,7.7397422)
\curveto(74.37476971,7.83872163)(74.3010119,7.91538324)(74.2078457,7.97166468)
\curveto(74.1146795,8.02697233)(74.00792842,8.065795)(73.88758254,8.08713903)
\curveto(73.76723665,8.10848305)(73.64107719,8.119165)(73.51200598,8.119165)
\lineto(72.14654228,8.119165)
\lineto(72.14654228,6.57431681)
\lineto(73.62264271,6.57431681)
\closepath
\moveto(73.7080376,3.76409994)
\curveto(73.85069369,3.76409994)(73.98656231,3.7776834)(74.11660741,3.80582412)
\curveto(74.24665252,3.83396485)(74.36116496,3.88054808)(74.46112857,3.94459008)
\curveto(74.56011828,4.00960589)(74.63872571,4.0979133)(74.69792475,4.2095024)
\curveto(74.7571238,4.32012764)(74.78624134,4.46276891)(74.78624134,4.63549849)
\curveto(74.78624134,4.97513476)(74.69016339,5.21772926)(74.49800748,5.36328198)
\curveto(74.30585157,5.50787085)(74.05158486,5.58064721)(73.73618125,5.58064721)
\lineto(72.14653234,5.58064721)
\lineto(72.14653234,3.76409994)
\lineto(73.7080376,3.76409994)
\closepath
}
}
{
\newrgbcolor{curcolor}{1 1 1}
\pscustom[linestyle=none,fillstyle=solid,fillcolor=curcolor]
{
\newpath
\moveto(76.5030236,9.25257435)
\lineto(78.13633889,9.25257435)
\lineto(79.68717103,6.63739495)
\lineto(81.22829401,9.25257435)
\lineto(82.85191008,9.25257435)
\lineto(80.39368387,5.17211982)
\lineto(80.39368387,2.63069058)
\lineto(78.93408007,2.63069058)
\lineto(78.93408007,5.2089949)
\lineto(76.5030236,9.25257435)
\closepath
}
}
{
\newrgbcolor{curcolor}{1 1 1}
\pscustom[linestyle=none,fillstyle=solid,fillcolor=curcolor]
{
\newpath
\moveto(34.25408708,22.39392648)
\curveto(34.25892764,14.8642521)(28.15752292,8.75717503)(20.627561,8.75230894)
\curveto(13.09759908,8.7474949)(6.98890748,14.84778895)(6.98453535,22.37741127)
\lineto(6.98453535,22.39392648)
\curveto(6.98016323,29.92360085)(13.08109951,36.03067793)(20.61106143,36.03506694)
\curveto(28.14200361,36.03988099)(34.24971495,29.93958694)(34.25408708,22.40991257)
\lineto(34.25408708,22.39392648)
\closepath
}
}
{
\newrgbcolor{curcolor}{0 0 0}
\pscustom[linestyle=none,fillstyle=solid,fillcolor=curcolor]
{
\newpath
\moveto(31.70315158,33.49848285)
\curveto(34.72669298,30.47525543)(36.23870224,26.77277961)(36.23870224,22.39392648)
\curveto(36.23870224,18.0141192)(34.75289971,14.35141345)(31.78129464,11.40584391)
\curveto(28.62771721,8.30351873)(24.90055861,6.75285924)(20.59987087,6.75285924)
\curveto(16.35112815,6.75285924)(12.68851911,8.29091548)(9.61305872,11.36698461)
\curveto(6.53712121,14.44262871)(4.99938234,18.11793772)(4.99938234,22.39392648)
\curveto(4.99938234,26.6689611)(6.53712121,30.37047411)(9.61305872,33.49848285)
\curveto(12.6103934,36.5226644)(16.27298509,38.03451665)(20.59987087,38.03451665)
\curveto(24.97919614,38.03452532)(28.67966223,36.52267308)(31.70315158,33.49848285)
\closepath
\moveto(11.64864732,31.46458022)
\curveto(9.09241149,28.8828829)(7.81477069,25.85870135)(7.81477069,22.38959817)
\curveto(7.81477069,18.92144913)(9.07980693,15.92294247)(11.60935891,13.39413024)
\curveto(14.13937934,10.86437254)(17.1512791,9.59996209)(20.64599506,9.59996209)
\curveto(24.14071102,9.59996209)(27.17828833,10.87696712)(29.76025377,13.43246043)
\curveto(32.21170607,15.80604364)(33.437905,18.79041176)(33.437905,22.38959817)
\curveto(33.437905,25.96156573)(32.19179729,28.99347577)(29.70105659,31.48395782)
\curveto(27.210793,33.9739628)(24.19260398,35.21946404)(20.64599506,35.21946404)
\curveto(17.09936879,35.21945537)(14.0996225,33.96765684)(11.64864732,31.46458022)
\closepath
\moveto(18.37555221,23.9154762)
\curveto(17.98493232,24.76698925)(17.40023866,25.19301034)(16.62043891,25.19301034)
\curveto(15.24185754,25.19301034)(14.55283144,24.2648369)(14.55283144,22.40944417)
\curveto(14.55283144,20.55362642)(15.24185754,19.62640712)(16.62043891,19.62640712)
\curveto(17.53076047,19.62640712)(18.18097529,20.0781031)(18.57111806,20.98345537)
\lineto(20.48197964,19.96602698)
\curveto(19.57118096,18.34791884)(18.20475306,17.5386566)(16.38268726,17.5386566)
\curveto(14.97744807,17.5386566)(13.85168657,17.96949173)(13.00638302,18.83021652)
\curveto(12.15963077,19.69189545)(11.7374778,20.87963685)(11.7374778,22.39392648)
\curveto(11.7374778,23.88153503)(12.17320691,25.06250208)(13.0451943,25.93725265)
\curveto(13.9171817,26.81205526)(15.00312563,27.24924839)(16.30503866,27.24924839)
\curveto(18.23095978,27.24924839)(19.60999225,26.4903818)(20.44364547,24.97419259)
\lineto(18.37555221,23.9154762)
\closepath
\moveto(27.36562175,23.9154762)
\curveto(26.97449872,24.76698925)(26.40145538,25.19301034)(25.64590185,25.19301034)
\curveto(24.2396824,25.19301034)(23.5360652,24.2648369)(23.5360652,22.40944417)
\curveto(23.5360652,20.55362642)(24.2396824,19.62640712)(25.64590185,19.62640712)
\curveto(26.55770681,19.62640712)(27.19624528,20.0781031)(27.56068446,20.98345537)
\lineto(29.51426968,19.96602698)
\curveto(28.60490236,18.34791884)(27.24040027,17.5386566)(25.42172633,17.5386566)
\curveto(24.01841296,17.5386566)(22.8950891,17.96949173)(22.05027134,18.83021652)
\curveto(21.2069283,19.69189545)(20.78430689,20.87963685)(20.78430689,22.39392648)
\curveto(20.78430689,23.88153503)(21.21323492,25.06250208)(22.07065724,25.93725265)
\curveto(22.92760244,26.81205526)(24.01842164,27.24924839)(25.34411244,27.24924839)
\curveto(27.26662435,27.24924839)(28.64372232,26.4903818)(29.47446079,24.97419259)
\lineto(27.36562175,23.9154762)
\closepath
}
}
{
\newrgbcolor{curcolor}{1 1 1}
\pscustom[linestyle=none,fillstyle=solid,fillcolor=curcolor]
{
\newpath
\moveto(87.0065449,26.78345945)
\curveto(87.0065449,20.85289916)(82.19837656,16.04523013)(76.26720034,16.04523013)
\curveto(70.33602411,16.04523013)(65.52785577,20.85289916)(65.52785577,26.78345945)
\curveto(65.52785577,32.71401974)(70.33602411,37.52168877)(76.26720034,37.52168877)
\curveto(82.19837656,37.52168877)(87.0065449,32.71401974)(87.0065449,26.78345945)
\closepath
}
}
{
\newrgbcolor{curcolor}{0 0 0}
\pscustom[linestyle=none,fillstyle=solid,fillcolor=curcolor]
{
\newpath
\moveto(79.37467918,29.89013802)
\curveto(79.37467918,30.30400141)(79.03889328,30.63927547)(78.62643782,30.63927547)
\lineto(73.88272299,30.63927547)
\curveto(73.47026752,30.63927547)(73.13448163,30.30401135)(73.13448163,29.89013802)
\lineto(73.13448163,25.14740271)
\lineto(74.45724292,25.14740271)
\lineto(74.45724292,19.53132124)
\lineto(78.05190795,19.53132124)
\lineto(78.05190795,25.14740271)
\lineto(79.37466924,25.14740271)
\lineto(79.37466924,29.89013802)
\lineto(79.37467918,29.89013802)
\closepath
}
}
{
\newrgbcolor{curcolor}{0 0 0}
\pscustom[linestyle=none,fillstyle=solid,fillcolor=curcolor]
{
\newpath
\moveto(77.87722624,32.88570367)
\curveto(77.87722624,31.98963621)(77.1507446,31.26323001)(76.25458408,31.26323001)
\curveto(75.35842356,31.26323001)(74.63194192,31.98963621)(74.63194192,32.88570367)
\curveto(74.63194192,33.78177112)(75.35842356,34.50817732)(76.25458408,34.50817732)
\curveto(77.1507446,34.50817732)(77.87722624,33.78177112)(77.87722624,32.88570367)
\closepath
}
}
{
\newrgbcolor{curcolor}{0 0 0}
\pscustom[linestyle=none,fillstyle=solid,fillcolor=curcolor]
{
\newpath
\moveto(76.24003157,38.6143385)
\curveto(73.0277405,38.6143385)(70.30845017,37.49354875)(68.08313449,35.25196925)
\curveto(65.79959366,32.93324124)(64.65830523,30.18852707)(64.65830523,27.01974452)
\curveto(64.65830523,23.85096198)(65.79959366,21.12565418)(68.08313449,18.84526194)
\curveto(70.36667532,16.5653566)(73.08596564,15.42516048)(76.24003157,15.42516048)
\curveto(79.43291426,15.42516048)(82.20073051,16.57409095)(84.54153253,18.87437646)
\curveto(86.74743983,21.05773686)(87.85087546,23.77285955)(87.85087546,27.01975446)
\curveto(87.85087546,30.26664937)(86.72803144,33.01039968)(84.48427134,35.25197918)
\curveto(82.23857338,37.49354875)(79.49113941,38.6143385)(76.24003157,38.6143385)
\closepath
\moveto(76.26913918,36.52801987)
\curveto(78.90108681,36.52801987)(81.13610171,35.59985297)(82.97419382,33.74448303)
\curveto(84.83169431,31.90900636)(85.76044456,29.66742686)(85.76044456,27.01974452)
\curveto(85.76044456,24.35314272)(84.8511027,22.14019084)(83.03241897,20.38185273)
\curveto(81.11571943,18.48767006)(78.86226011,17.54106065)(76.26913918,17.54106065)
\curveto(73.67504435,17.54106065)(71.44001951,18.47796191)(69.56408453,20.35273822)
\curveto(67.68620176,22.22751452)(66.74871625,24.44968766)(66.74871625,27.01973459)
\curveto(66.74871625,29.59026841)(67.69687488,31.83137095)(69.59319213,33.7444731)
\curveto(71.4118858,35.59985297)(73.63719154,36.52801987)(76.26913918,36.52801987)
\closepath
}
}
\end{pspicture}
 \\ {\tt cc-by}

        \column{0.5\textwidth}
        Specify licence by:
\begin{lstlisting}[language=TeX]
\def\licence{cc-by-nc-sa}
\end{lstlisting}

    or omit the command entirely if you do not want to specify one.

    \end{columns}

%    If you do not want to specify licence, do not \lstinline{\def\licence{}}.

\end{frame}

% -----------------------------------------------------------------------------

\begin{frame}[fragile]
    \frametitle{Sections}

    Sections have a title slide. To disable the title slide do:

    \begin{lstlisting}[language=TeX]
        \AtBeginSection{}
    \end{lstlisting}

    You can generate an outline of the slides by:

    \begin{lstlisting}[language=TeX]
        \outline{Title of your outline slide}
        \outlinecurrent{Outline with highlighted current section}
    \end{lstlisting}

\end{frame}


% -----------------------------------------------------------------------------

\begin{frame}[fragile]
    \frametitle{Equations}

    \centering Serif font also for equations

    $$i\hbar\frac{\partial}{\partial t} \Psi(\mathbf{r},t) = \left [ \frac{-\hbar^2}{2\mu}\nabla^2 + V(\mathbf{r},t)\right ] \Psi(\mathbf{r},t) ]$$

    \begin{lstlisting}[language=TeX]
$$ i\hbar\frac{\partial}{\partial t} \Psi(\mathbf{r},t) = \left [ \frac{-\hbar^2}{2\mu}\nabla^2 + V(\mathbf{r},t)\right ] \Psi(\mathbf{r},t) ]$$
    \end{lstlisting}

    Equations may break compilation with Mik\TeX. If this is your case use the
    package with option \lstinline[language=TeX]{miktex}.

\begin{lstlisting}[language=TeX]
    \usepackage[miktex]{ufallslides}
\end{lstlisting}

\end{frame}

% -----------------------------------------------------------------------------

\begin{frame}
    \frametitle{Color Palette}

    Named colors from a pallete from
    \url{https://www.colorcombos.com/colors/FF6600} \\[15pt]

    \begin{center}
    \scalebox{1.2}{%
    \begin{tabular}{cc}

    \colorbox{ufal}{\bf ufal} &     \color{ufal} \bf ufal \\
    \colorbox{ufalviolet}{\bf ufalviolet} &     \color{ufalviolet} \bf ufalviolet \\
    \colorbox{ufalblue}{\bf ufalblue} &     \color{ufalblue} \bf ufalblue \\
    \colorbox{ufalred}{\bf ufalred} &     \color{ufalred} \bf ufalred \\
    \colorbox{ufallightblue}{\bf ufallightblue} &     \color{ufallightblue} \bf ufallightblue \\
    \colorbox{ufalgreen}{\bf ufalgreen} &     \color{ufalgreen} \bf ufalgreen \\
    \end{tabular}}
    \end{center}

\end{frame}


% -----------------------------------------------------------------------------

\begin{frame}[fragile]
    \frametitle{Labels from the web}

    \slidesbox{Slides}
    \readingbox{Reading}
    \hwbox{Homework}
    \questionbox{Question}
    \timebox{1 h}
    \calendarbox{Oct 15}
    \pointsbox{100 points}

    \begin{lstlisting}[language=TeX]
\slidesbox{Slides}
\readingbox{Reading}
\hwbox{Homework}
\questionbox{Question}
\timebox{1 h}
\calendarbox{Oct 15}
\pointsbox{100 points}
\slidesbox{Slides}
    \end{lstlisting}
\end{frame}

% -----------------------------------------------------------------------------
\section{Another section}
% -----------------------------------------------------------------------------

\begin{frame}[fragile]
    \frametitle{Code listings}

    This code snippet:
    \begin{lstlisting}[language=Python]
print("Hello, ÚFAL.")
x = 3 + 5
    \end{lstlisting}
%
    is produced by putting the code between {\tt  \textbackslash
    begin\{lstlisting\}} and {\tt \textbackslash end\{lstlisting\}}

    Inline code (\lstinline[language=Python]{import numpy as np}) can be
    inserted with the \lstinline[language=TeX]{\lstinline} command.

    Do not forget to start the frame with \lstinline[language=TeX]{fragile}
    option to beginning of the frame.

\end{frame}

% -----------------------------------------------------------------------------

\begin{frame}[fragile]
    \frametitle{References}

    Full citation on slide: \\ {\tiny \bibentry{helcl2017neural}}

\begin{lstlisting}[language=TeX]
Full citation: \\ {\tiny \bibentry{helcl2017neural}}
\end{lstlisting}

    \citet{sennrich2016neural} uses attention \citep{bahdanau2015neural}.

    \begin{lstlisting}[language=TeX]
\citet{sennrich2016neural} uses attention \citep{bahdanau2015neural}.
    \end{lstlisting}

    \tiny
    \citet{snover2006study,lu2016knowing,tu2016modeling,feng2016improving,zhang2016recurrent,alkhouli2016alignment,graves2014neural,specia2016shared,elliott2016multi30k}

If you prefere managing bibliography on your own, use the package with option
    \lstinline[language=TeX]{custombibset}.

\end{frame}

% -----------------------------------------------------------------------------

\begin{frame}[fragile]
    \frametitle{Summary, outline, references}

    The summary slide can be inserted by calling:

    \begin{lstlisting}[language=TeX]
\summary{Name of the summary slide}{%
    Content of the summary slide
}
    \end{lstlisting}

    Outline with optionally highlighted current section can be inserted by:
    \begin{lstlisting}[language=TeX]
\outline{Outline slide title}
\outlinecurrent{Whatever outline title you wish}
    \end{lstlisting}

    To show the references do:

    \begin{lstlisting}[language=TeX]
\references{pathToYourBibFile.bib}
    \end{lstlisting}

\end{frame}

% -----------------------------------------------------------------------------

\begin{frame}
    \frametitle{Itemize}

    \begin{itemize}[<+->]

        \item All human beings are born free and equal in dignity and rights.

        \item They are endowed with reason and conscience and should act
            towards one another in a spirit of brotherhood.

        \item Everyone is entitled to all the rights and freedoms set forth in
            this Declaration, without distinction of any kind, such as race,
            colour, sex, language, religion, political or other opinion,
            national or social origin, property, birth or other status.

    \end{itemize}

\end{frame}

% -----------------------------------------------------------------------------

\begin{frame}
    \frametitle{Enumerate}

    \begin{enumerate}[<+->]

        \item All human beings are born free and equal in dignity and rights.

        \item They are endowed with reason and conscience and should act
            towards one another in a spirit of brotherhood.

        \item Everyone is entitled to all the rights and freedoms set forth in
            this Declaration, without distinction of any kind, such as race,
            colour, sex, language, religion, political or other opinion,
            national or social origin, property, birth or other status.

    \end{enumerate}

\end{frame}

% -----------------------------------------------------------------------------

\begin{frame}[allowframebreaks]
    \frametitle{What happens with too much content?}

    Babakotia, an extinct genus of sloth lemurs, lived in the northern part of
    Madagascar. The name comes from the Malagasy word for the indri, to which
    all sloth lemurs are closely related. Its morphological traits show
    intermediate stages between the slow-moving smaller sloth lemurs and the
    suspensory large sloth lemurs, and suggest a close relationship between
    both groups and the extinct monkey lemurs. All sloth lemurs share many
    traits with living sloths, demonstrating convergent evolution. Babakotia
    had long forearms, curved digits, and highly mobile hip and ankle joints.
    It shared its range with other sloth lemurs, including Palaeopropithecus
    ingens and Mesopropithecus dolichobrachion. It was primarily a leaf-eater,
    though it also ate fruit and hard seeds. It is known only from subfossil
    remains and may have died out shortly after the arrival of humans on the
    island, but not enough radiocarbon dating has been done with this species
    to know for certain. Babakotia radofilai is the sole member of the genus
    Babakotia and belongs to the family Palaeopropithecidae, which includes
    three other genera of sloth lemurs: Palaeopropithecus, Archaeoindris, and
    Mesopropithecus. This family in turn belongs to the infraorder
    Lemuriformes, which includes all the Malagasy lemurs.[5][2]

\end{frame}


% -----------------------------------------------------------------------------

\summary{Summary}{%
    \begin{enumerate}

        \item This template is tremendous.

        \item If you don't use the template you will be very very sad.

        \item Believe me. It's tremendous.

    \end{enumerate}
}
% -----------------------------------------------------------------------------

\references{references.bib}

\end{document}
